%(IoTの背景
IoTとは,様々なモノにコンピュータを取り付け,インターネットを介して相互に情報をやり取りすることで,様々な自動化を図ろうという概念である.
IoTサービスとは,IoTによる利便性をユーザーに提供するもので,IoT機器とサーバーのプログラムがインターネットを介して通信し合うことで成り立っている.
そのため,IoTサービスの円滑な提供のためには,IoT機器の監視は不可欠である.
\medskip

%(2章 問題として何を取り上げたのか
しかし,IoT機器が多量に存在することや,IoT機器が接続するネットワークが多様であることから,IoT機器の監視には技術的困難がある.
IoTサービスでは,多量のIoT機器を使用するため,個々のIoT機器を識別し,適切に管理することは難しい.
また,IoT機器が接続されるネットワークを予測することは困難である.
そのため,機器のIPアドレスを使用した既存手法を適応することは現実的ではない.
\medskip

%(第3章 提案として何を提案したのか
そこで,本研究では,IoT機器から通知を送ることで,IoT機器が接続されるネットワークによらない機器の監視を可能にすることを提案した.
また,機器の監視に必要な要件を抽出し,機器の監視に特化した監視サービスを提案した.
%(第4章 実装として何を実装したのか
この機器監視サービスを実現する為,IoT機器にインストールされるエージェントプログラム,機器監視サーバ上で動作するエージェントプログラム用インターフェース「かおりちゃん」,Webアプリケーションを作成した.
\medskip

%(第5章 
必要な機能を持っているのか検証を行い,オープンソース利用の既存監視手法と手順について比較した.
その結果,IoT機器の動作状態について監視できていることを確認した.
また,既存監視手法との比較では,手間がかからなくなった事を確認した.
今後の課題として,IoT機器への設定の簡略化や,ユーザインターフェースの向上,様々なIoT機器への対応があることがわかった.

