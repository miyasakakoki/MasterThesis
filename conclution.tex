本研究では、IoTサービスの開発にかかるコストを削減するために、IoT機器から通知に基づいた機器監視を行う手法を提案し、それらを用いたIoT機器監視サービスの開発を行った。
本論文での実験及び考察において得られた知見を以下に示す。
\begin{itemize}
	\item NAPT環境下にある機器の状態を監視するために、機器から状態を定期的に通知する手法は有効であることが分かった。
	\item また、既存の機器監視サービスの多くは、機器のIPアドレスをサーバー側で管理しなければならないが、本手法を用いることで、IPアドレスに関係なく監視できることが分かった。
\end{itemize}

また、今後の課題としては、次のような物が上げられる。
\begin{itemize}
	\item ユーザビリティーの向上\\
		現状では、デバイスIDやサーバーIPアドレスをIoT機器に打ち込まなければならず、また、自動起動の設定も行わなければならない。\\
		シェリスクリプトを実行すれば自動起動の設定まで行われる等、いっそうの簡略化を行う必要がある。
	\item デバイスの識別問題の解決\\
		現状では、どのデバイスが何だったのかまでは管理できていない。\\
		QRコードを出力し、デバイスに貼り付けるなどによって、簡略化できると思われる。
	\item アラート機能の実装\\
		アラートメールの送信などもできれば、常に監視していなくて良くなるのではないかと思われる。
\end{itemize}


