
%(IoTの背景
IoTとは,様々なモノにコンピュータを取り付け,インターネットを介して相互に情報をやり取りすることで,様々な自動化を図ろうという概念である.
IoTサービスとは,IoTによる利便性をユーザーに提供するもので,IoT機器とサーバーのプログラムがインターネットを介して通信し合うことで成り立っている.
IoTサービスの円滑な提供のためには,この構造を維持しなければならない。
そのため、IoT機器の監視が不可欠である。
\medskip

本論文では、そのIoT機器の監視がサービス提供者にとって負担となっている事を取り上げた。
株式会社ルナネクサスが提供している太陽光発電の発電量の監視のためのIoTサービスを主にとりあげ、聞き取りを行った。
その中でIoT機器の監視が負担である事を問題として捉え、分析した。
また、岡本商店街にて行った実験から、監視における技術的課題を明確にした。

%(2章 問題として何を取り上げたのか
IoT機器の監視には、次のような技術的課題があることが分かった。
\begin{itemize}
\item ネットワークの監視はIoTサービスが提供する機能とは別の機能である事
\item IoT機器の接続されるネットワークは想定できない事
\item IoT機器の接続されるネットワークがプライベートアドレスである場合が多い事
\item IoT機器は大量に利用される為、サーバへ設定することが負担となること
\end{itemize}
\medskip

%(第3章 提案として何を提案したのか
そこで,本研究では,IoT機器から通知を送ることで,IoT機器が接続されるネットワークによらない機器の監視を実現し、IoTサービスとは別個の機器監視サービスとして開発することで、提供者の負担を軽減することを提案した。
既存手法では何が足りないのか議論し、IoTサービスの実践やサービス提供者への聞き取りを通して、サービスに必要な要件を抽出し、設計を行った。
%(第4章 実装として何を実装したのか
この機器監視サービスを実現する為,IoT機器にインストールされるエージェントプログラム,機器監視サーバ上で動作するエージェントプログラム用インターフェース,Webアプリケーションを作成した.
\medskip

%(第5章 
また、要件を満たしているか検証を行い、IoT機器の動作状態について監視できていることを確認した。
今後の課題として,IoT機器への設定の簡略化や,ユーザインターフェースの向上,様々なIoT機器への対応があることがわかった.
本来ならば株式会社ルナネクサスにて使用してもらい評価を得るべきところではあるが、スケジュールの都合により、実現しなかった。
しかし、今後IoTサービスの普及がより進み、使用するIoT機器の数が多くなることから、本サービスが求められることが推測される。

