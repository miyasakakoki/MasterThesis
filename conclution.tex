よって、このアプローチは有効であるということが分かった。


今後の課題としては、次のような物があることが分かった。
\begin{itemize}
	\item ユーザビリティーの向上\\
		現状では、デバイスIDやサーバーIPアドレスをIoT機器に打ち込まなければならず、また、自動起動の設定も行わなければならない。\\
		シェリスクリプトを実行すれば自動起動の設定まで行われる等、いっそうの簡略化を行う必要がある。
	\item デバイスの識別問題の解決\\
		現状では、どのデバイスが何だったのかまでは管理できていない。\\
		QRコードを出力し、デバイスに貼り付けるなどによって、簡略化できると思われる。
	\item アラート機能の実装\\
		アラートメールの送信などもできれば、常に監視していなくて良くなるのではないかと思われる。
\end{itemize}


