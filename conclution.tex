本研究の背景には、IoTサービスにおける機器監視の煩わしさが挙げられる。
具体的には、
\begin{enumerate}
	\item IoTサービスに組み込む形で機器監視を行った場合、IoTサービス毎に監視部分を開発するので、コストが高い
	\item 設置箇所が遠く、分散して設置されるため、定期的に確認しに行くことは現実的ではない
\end{enumerate}
が挙げられる。\\

そこで、機器監視をサービスから独立させ、IoT機器監視サービスとして提供すれば良いのではないかと考え、開発を行った。
しかし、既存手法を調査する中で、IoT機器はNAPT環境下に置かれる事があることがわかり、IoT機器から定期的に通知を送ることで解決を図った。

プロトタイプの開発の結果、下記のような知見を得ることが出来た。
\begin{itemize}
	\item NAPT環境下にある機器の状態を監視するために、機器から状態を定期的に通知する手法は有効であることが分かった。
	\item また、既存の機器監視サービスの多くは、機器のIPアドレスをサーバー側で管理しなければならないが、本手法を用いることで、IPアドレスに関係なく監視できることが分かった。
\end{itemize}

また、今後の課題としては、次のような事があることが分かった。
\begin{itemize}
	\item ユーザビリティーの向上\\
		現状では、デバイスIDやサーバーIPアドレスをIoT機器に打ち込まなければならず、また、自動起動の設定も行わなければならない。\\
		シェリスクリプトを実行すれば自動起動の設定まで行われる等、いっそうの簡略化を行う必要がある。
	\item デバイスの識別問題の解決\\
		現状では、どのデバイスが何だったのかまでは管理できていない。\\
		QRコードを出力し、デバイスに貼り付けるなどによって、簡略化できると思われる。
	\item アラート機能の実装\\
		アラートメールの送信などもできれば、常に監視していなくて良くなるのではないかと思われる。
\end{itemize}
これらの課題を解決することで、IoTサービスの開発の手間を省けるのではないかと考える。


