まず、要件を抽出するために、以下のような実験を行った。

%岡本商店街での事例
\section 岡本商店街での事例
岡本商店街とは、兵庫県東灘区にある阪急岡本駅とJR摂津本山駅の間にある商店街の事である。
目的は、岡本商店街内を往来する人の流れを観測し、商店街の活性化につなげる事であったが、
有用な知見(?)が得られたので、事例として上げることにする。


スマートフォンのWifi接続機能をONにした時、スマートフォンから定期的にプローブパケットというものが、送信される。
RaspberryPiとampsenceを用いて、そのプローブパケットを観測し、可視化することを行った。
RaspberryPiは、商店に設置させてもらい、インターネットとの接続は、商店に敷設済みのネットワークを使用させてもらう他、携帯電話回線網を使用した。

%課題と考察


%学内でのwifi人流観測
\section 学内でのwifiを用いた人流観測
学内にて、
wifiを用いた人流観測を行った。
構成としては、先の事例での構成とほぼ同じだが、先の事例での失敗点等を踏まえて、実施した。

%課題と考察


\section 要件の抽出
上記の実験から、以下のような機能が必要となることが分かった。

\begin{itemize}
\item IoTデバイスの稼働状態がわかる\\
	IoTデバイスの稼動状態は、稼働している・稼働していない・ネットワークに接続されていないの3つ必要である。
\item IoTデバイスの稼働状態の記録を閲覧することができる\\
	データの分析を行う際に、それら稼働状況の記録が必要になる事が分かった。\\
	また、それらの記録が時刻と共に、整理されている必要があることも分かった。
\end{itemize}

そこで、IoTサービスとは独立した、IoTデバイスの稼動状態を監視・管理することを簡単にするサービスを開発し、提供すれば良いのではないかと考えた。
何故ならば既存手法では上記に述べたとおり、簡単には解決できないからだ。




