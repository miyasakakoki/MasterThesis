\section{IoT機器の監視とは}
%(ここにおける監視とはどういったものなのか、どのような状態を監視しなければならないのか、また、監視において何をしなければならないのか、何が起きたらどう対処しなければならないのか、何故そうしなければならないのか)
IoT機器の監視とは、動作の状態・通信の状態を監視することである。
動作状態とは、少なくとも電源が入っているのかどうか、(?)
通信状態とは、インターネットに接続され、サーバーと通信ができているのかどうかである。

このように、個々のIoT機器の動作状態、通信状態を確認することが、IoT機器の監視である。

\section{IoT機器の監視に必要な機能とは}
%(どのような機能が必要となるのか、IoT機器の状態を一覧して表示することとはどういったことなのか、IoT機器の過去の状態を監視することができるとはどういったことなのか、何故そのような機能が必要なのか)
IoT機器の監視とは、個々のIoT機器の動作状態、通信状態を確認することである。
そのため、次のような機能が必要となる。
\begin{itemize}
	\item IoT機器の動作状態を一覧して表示する機能
	\item IoT機器の過去の状態を確認する機能
\end{itemize}

IoT機器の動作状態を一覧して表示する機能とは、現在のIoT機器の動作状態を見ることができる機能である。
あるIoT機器が動作しなくなった時に、動作していないことを明確に示す必要がある。

IoT機器の過去の状態を確認する機能とは、過去のIoT機器の動作状態と通信状態を確認することができる機能である。
過去の動作状態や通信状態を整理し、いつ動作していて、いつ動作していなかったのか、各IoT機器ごとに示す必要がある。

このように、IoT機器の監視には、IoT機器の動作状態を一覧して表示する機能、IoT機器の過去の状態を確認する機能を持つ必要がある。

\section{IoT機器の監視に求められる条件とは}
%(どのような制約条件が存在するのか、何故そのような制約が発生するのか)
IoT機器の監視には、上記機能が必要である他に、IoT機器が設置されるネットワークにかかわらず監視ができることが求められる。
何故ならば、IoT機器が設置されるネットワーク環境は多様であるからである。

IoT機器が設置されるネットワーク環境として、次のような環境が挙げられる。
\begin{itemize}
	\item IoT機器に対しプライベートアドレスが与えられる環境
	\item IoT機器とサーバーの通信が制限される環境
\end{itemize}

多くの場合、IPアドレスの節約の観点から、IoT機器に対し、プライベートアドレスのみ与えられる。
この場合、インターネット側からは、IPアドレスを指定することが出来ないため、通信は常にIoT機器から始まる必要がある。
また、IoT機器とサーバーとの通信がセキュリティの観点から制限される場合もある。
HTTP等一般的に広く使用される通信を除いてブロックされる事がある。

このようにIoT機器の監視には、IoT機器から通信が始まることや、HTTP等頻繁に利用される通信を利用しなければならないといった条件がある。

\section{IoT機器監視に求められる要件とは}
%(機器監視に求められる要件を整理し、述べる)
これらを踏まえて、IoT機器の監視には次のような機能要件が求められる。
\begin{itemize}
	\item IoT機器の動作状態を一覧して表示する機能
	\item IoT機器の過去の状態を確認する機能
\end{itemize}
また、非機能要件として、次のような制約がある。
\begin{itemize}
	\item プライベートアドレスが与えられたとしても動作する事
	\item インターネットの通信として一般に広く利用されているHTTP等の通信を用いる事
\end{itemize}

