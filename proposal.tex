%4章提案
\section{IoT機器監視サービスの提案}
IoT機器からの通知に基づいた、設定不要の独立した状態監視のためのサービスを提案する。
機器から通知を送ることで、IoT機器の接続されるネットワークが、プライベートアドレスを使用していても監視可能であること、ネットワークが違っていても一つの画面で確認できることを満たす。
また、状態監視の為のサービスを独立させることで、IoTサービスの変更が不要であることや、監視サーバを立てる必要が無いことを満たす。
IoT機器への設定の必要性については、あらかじめ、IoT機器自体に存在する識別IDを用いることとする。
こうすることで、監視サーバに対して、設定をする必要がなくなる。



このように、今後使用するIoT機器が増えていくことを考えると、現状の手動での監視は負担となる。
そこで、新規にIoT機器の監視に汎用的に使用できるシステムを開発し、サービスとして提供することで、問題の解決が図れるのではないかと考えた。
実験と聞き取りから得られた要件を以下にまとめる。
\begin{itemize}
\item 機器が起動し動作していることが確認できること
\item CPUの温度等も確認できると良い
\item 各機器に対する監視に関わる設定の簡略化もできたら良い
\item 機器の異常をメールなどで知らせることができたら良い
\item 機器の状態をIoTサービスごとに管理したい
\end{itemize}


