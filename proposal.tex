まず、要件を抽出するために、以下のような実験を行った。
%学内でのwifi人流観測
%岡本商店街での事例

上記の実験から、以下のような機能が必要となることが分かった。

\begin{itemize}
\item IoTデバイスの稼働状態がわかる\\
	IoTデバイスの稼動状態は、稼働している・稼働していない・ネットワークに接続されていないの3つ必要である。
\item IoTデバイスの稼働状態の記録を閲覧することができる\\
	データの分析を行う際に、それら稼働状況の記録が必要になる事が分かった。\\
	また、それらの記録が時刻と共に、整理されている必要があることも分かった。
\end{itemize}

そこで、IoTサービスとは独立した、IoTデバイスの稼動状態を監視・管理することを簡単にするサービスを開発し、提供すれば良いのではないかと考えた。
何故ならば既存手法では上記に述べたとおり、簡単には解決できないからだ。




