%	背景
 半導体技術の進歩 -> コンピューターの小型化・低価格化 1
 インターネット回線網の普及 2
 1&2 -> IoTが注目されている

 IoTとは
 Internet of Things
 様々な物がインターネットにつながり、相互に情報を交換し合うことで、様々な自動化を実現しよう という考え方。
 IoTの例:
1.
2.
3.

 また、IoTが注目される事によって、その自動化で収益を得ようとするサービスが登場し始めた。
 IoTサービスの例:
 1.餌やりの自動化
 2.
 3.

%問題
 しかし、IoTサービスを開発・運用するには、様々な問題がある。
 1.開発が大変とか?
 2.故障検知の問題とか?(死活監視)
	死活監視が何故必要か->サービスを止めないために必要
	死活監視が
 3.設置場所の問題とか?

その中で、私は、デバイスの故障検知?の問題に目をつけた。
何故そこに目をつけたのか
問題の詳細

このように、デバイスの故障検知?には、上記のような問題が複雑に絡み合っており、単純には解決できない。

% 目的
そこで、IoTサービスとは独立したIoTデバイスの監視サービスを開発することにより、デバイスの故障検知に係る問題の解決を図ることにした。
システムの構築に先立って、どのような機能が必要となるのか実験し、次のような機能が必要になることが分かった。
1.
2.
3.
また、必須ではないものの、次のような機能があると、嬉しい事が分かった。
1.
2.
3.
%	構成
これら必要な機能を踏まえ、次のようなシステムを構築
%	検証と結果
システムを構築し協力を得て、検証を行った所、次のような結果が得られた。

これら結果により、システムの有効性が立証できた。
%	今後の課題
今後の課題としては、
1.
2.
3.
