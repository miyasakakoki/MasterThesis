近年,半導体技術の進歩により,コンピューターの小型化・低価格化が進んでいる.
また,インターネット回線網の普及もあり,Internet of Thingsという概念が注目され,それによって収益を得るIoTサービスが登場してきた.
Internet of Things(IoT)とは,様々な物がインターネットにつながり,相互に情報を交換し合うことで,様々な自動化を実現する概念である.

しかし,IoTサービスを開発・運用するには,開発コストの問題・セキュリティーの問題・稼働率の問題など様々な問題がある.

そこで,本研究では,IoTデバイスの死活監視問題に焦点を当て,IoTサービスとは独立したIoTデバイスの監視サービスを開発することにより,デバイスの故障検知に係る問題の解決を図ることにした.
システムの構築に先立って,どのような機能が必要となるのか,実験し,
デバイスの電源の状態(電源が入っているのか・入っていないのか)・ネットワークの状態(インターネットへ接続されているのかいないのか)が時系列に沿って整理されている事で,対処が決まる事が分かった.
そこで,上記必要な機能を実装したシステムを提供し,協力者の理解を得て検証し評価を得た.




