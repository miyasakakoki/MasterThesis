%(IoTとは何なのか)
近年,IoTが注目を集めている.
IoTとは,コンピュータをさまざまなモノに取り付けることで,利便性の向上を図る概念である.
近年の半導体技術の進歩により,コンピュータが安価・小型になったこと,インターネットへの通信が様々な場所で安価に行えるようになったことにより,安価に実現可能となった.
\medskip

%(IoTサービスとは何なのか,何故期待されているのかを説明)
それらのモノが連携して提供するサービスはIoTサービスと呼ばれ,より生活に身近なサービスの登場が期待されている.
IoTサービスは,IoT機器とサーバーがインターネットを介して通信し合うことで,成り立っている.
IoT機器は,モノにコンピュータが取り付けられた機器で,周囲の状況を検知,または,周囲へ働きかける機能を持つ.
サーバーは,IoT機器からの情報を蓄積・分析し,IoT機器へ指示を送るか,ユーザーへ分析結果を表示する機能を持つ.
これらIoT機器とサーバーが連携することで,IoTサービスは利便性をユーザーへ提供している.
\medskip

%(IoTサービスの提供にはどのような事が必要となるのか)
IoTサービスを円滑に提供するには,IoT機器とサーバーの連携を正常に維持しなければならない.
そのため,IoT機器の動作状態や通信状態の監視が重要となる.
%(IoT機器の動作状態や通信状態の監視が,何故困難なのか)
IoT機器の監視は,接続されるネットワークが多様であることから,通知型の監視を用いる必要がある.
そのため,多量に存在するIoT機器へ個別に設定をする必要がある.
また,多量のIoT機器を監視サーバへ登録する必要もある.
これら設定は,IoT機器の交換や追加の際にも発生するため,IoTサービスの開発・運用の負担となっている.
\medskip

%(それらを解決するために何が求められるのか)
本研究では,これら負担が,IoT機器と監視サーバに分散して設定が存在することが原因だと考えた.
%(何を提案するのか)
そこで,分散して存在する設定をサーバにて一元的に管理し,IoT機器への個別の設定,監視サービスへの登録作業を省力化することを提案する.
サーバからIoT機器へ個別のIDを付与し,機器への設定を簡略化し,IDを付与した時点で機器を登録することで,登録を自動化する.
機器は,個別のIDに紐付いたURLへhttpsにて通知することで,機器の監視を可能にした.

\begin{comment}
数も多く,さまざまなネットワークを介して接続されるIoT機器の監視は困難な問題である.
IoT機器が設置される様々なネットワークの構成を把握することは,IoT機器が多量であることを考えると現実的ではない.
また,従来の監視手法はパーソナルコンピュータを対象としたもので,利用しにくい.
現状としてはIoTサービス開発者が,IoTサービス毎に実装しているため負担が大きいといった問題がある.
%(それらを解決する為に,何が求められるのか)
そのため,設置されるネットワークに関係なく状態が監視できることが求められる.
また,IoT機器の状態を一覧して確認できることや,IoT機器の過去の動作状態や通信状態を確認できることが必要である.
IoTサービスの開発者の負担を減らすためにも,IoT機器の監視サービスが必要である.
\medskip

%(何を提案するのか)
そこで,我々は,IoT機器からの通知に基づいた機器監視サービスを提案する.
IoT機器が自身の過去の動作状態や通信状態を記録することで,設置されるネットワークに関係なく状態監視をすることを可能にする.
また,サービスを機器監視に特化させ独立させることで,IoTサービスに変更を与えること無く,機器を監視することを可能にする.
この仕組みを用いることで,IoT機器が設置されるネットワークに関係無く状態を監視することや,IoT機器の過去の状態や通信状態を確認することを容易にし,
IoTサービス開発者はサービスの開発に専念できる.
本研究では,IoT機器からの通知による機器の設置環境によらない機器の監視を行うことにより,IoTサービスの維持を容易にするシステムの開発に取り組む.
\medskip
\end{comment}

