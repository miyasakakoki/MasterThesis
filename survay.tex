IoT機器の監視は、次のような要件を満たしている必要があることが分かった。
\begin{itemize}
\item IoT機器の接続されるネットワークが、プライベートアドレスを使用していても、監視可能であること\\
	IoT機器が接続されるネットワークを予め予測することは難しい。
	そのため、IoT機器にプライベートアドレスが割り振られる可能性がある。
	そのような状況でも監視することができる必要がある。
\item ネットワークが違っていても、一つの画面で確認できること\\
\item IoTサービスの変更が不要であること\\
	サーバの変更は開発者への負担となるため、サーバの変更を避けたい。
\item 監視サーバを立てる必要がないこと\\
	新たに監視サーバを立てることは、設定や構築作業が必要となるため、開発者の負担となる。
	そのため、新規に監視サーバを立てる必要が無いことが求められる。
\item 機器に異常があった場合に、運用者に通知を行う機能があること\\
	常に監視画面を見ているわけには行かないので、アラート機能等があると良い。
\end{itemize}

そこで、従来からある手法にて対応できるか、考えた。
\section{サーバからの問い合わせによる監視}
	従来から機器監視に用いられてきた手法として,定期的に機器監視サーバから対象機器に状態を問い合わせる手法がある.
	機器監視サーバから,監視対象機器上のエージェントプログラムに,現在の状態を問い合わせる事で機器の監視を実現している.
	\medskip

	代表的なものとして、次のような物がある。
	\begin{itemize}
		\item Ping\\
			Pingとは、ICMPメッセージをやり取りするプログラムである。
			ICMPとは、InternetControlManagementProtocolの事で、IPネットワークにおいて、IPパケットの不到達等を通知する為の取り決めである。
			ICMPには、ICMP Echo Request Echo Replyが定義されており、ICMP Echo Requestを受け取った機器は、ICMP Echo Replyを返さなくてはならない。
			監視を行う機器から、Pingを用いて、ICMP Echo Requestを監視対象機器のIPアドレスに送信し、Echo Replyの有無から、機器の存在を確認する事ができる。
		\item SNMPによる問い合わせ\\
			SNMPとは、Simple Network Management Protocolの事で、ネットワークの監視・管理を行うための、取り決めである。
			SNMPには、SNMPマネージャとSNMPエージェントが存在する。
			SNMPマネージャは、監視を行う機器上で動作し、SNMPエージェントは、監視される機器の上で動作する。
			SNMPを利用した機器の状態監視の方法として、SNMPマネージャがSNMPエージェントに問い合わせを行い取得する方法がある。
	\end{itemize}
	
	この手法のメリットを以下にまとめる。
	\begin{itemize}
		\item 手軽に機器の状態やネットワークの状態を確認することができる事
		\item 見やすく表示する事ができるソフトウェアが豊富に存在する事
		\item コンピューターであれば、ほぼ全ての機器が対応している事
	\end{itemize}

	また、デメリットとして、次のような物がある。
	\begin{itemize}
		\item 監視する側のコンピュータに全ての監視対象のIPアドレスを設定する必要があること
		\item 攻撃の足がかりとして利用される事があるので、間にあるネットワーク機器によってブロックされてしまう事がある
	\end{itemize}

\begin{comment}
	IoT機器の監視は、機器がNAPTの内側にある場合が多く、ユーザーに近い場所に設置されることが多いので、この方法を用いるためには、
	機器がグローバルIPアドレスを持っていること、間のネットワークにて監視用パケットがブロックされない様ネットワーク機器を設定すること、
	が必要である。

	この形では,機器監視サーバは,監視対象機器のIPアドレスを覚えておかねばならない.
	そのため,監視対象が接続されるネットワークが切り替わった場合,監視サーバ上に記録されたIPアドレスを変更しなくては監視できない.
	監視対象が,多量且つ移動するIoT機器の監視において,頻繁に監視サーバ上に記録されたIPアドレスを書き換えるのは大変である.
	また,プライベートアドレスの利用時には,サーバーからIoT機器への到達性が失われる可能性がある.
\end{comment}

\section{監視対象機器からの通知による監視}
	機器監視手法として,定期的に監視対象機器から機器監視サーバーへ状態を送信するという手法がある.
	監視対象機器上のエージェントプログラムが,指定した時間毎に機器監視サーバーへ状態を送信することで,機器の監視を実現している.
	\medskip
	
	代表的なものとして、次のようなものがある。
	\begin{itemize}
		\item SNMP Trapによる取得\\
			SNMPでは、予めSNMPエージェントに設定をしておくことによって、状態の変化をSNMPマネージャへ知らせることができる機能がある。
			この機能を利用して、SNMPエージェントの状態を取得できる。
		\item サーバーのリソース監視ソリューション\\
			サーバのリソース監視ソリューションの中には、独自に開発されたエージェントプログラムを監視対象にインストールすることに寄って、監視を行うものがある。
			代表的なものとして、以下の2つをあげる。
			\begin{itemize}
				\item TeregrafとInfluxdbによる、機器の状態の通知と蓄積\\
					Teregrafとは、監視対象にインストールされるエージェントプログラムで、監視される機器の上で動作する。
					Influxdbとは、時系列データを格納することに特化したデータベースで、この中では機器の状態を蓄積することに使用され、監視を行う機器の上で動作する。
					Teregrafは、機器の状態を定期的に取得しInfluxdbに送信する。そして、Influxdbは、それを蓄積する。
					Influxdbに蓄積されたデータは、Influxdbに対応した可視化アプリケーション(Grafana等)によって可視化され、監視を行うことができる。
				\item Logstashによるログの転送による機器の監視\\
					Logstashとは、アプリケーションのログやシステムのログを転送するためのプログラムであり、監視される機器上で動作する。
					Logstashは、ログファイルの監視をしており、ログファイルに追記があった場合、追記分のデータをデータベースに送信することができる。
					データベースとしては、Elasticsearchが良く用いられる。
					データベースに蓄積されたデータは、対応した可視化アプリケーションによって可視化を行うことで、監視をする。
			\end{itemize}
	\end{itemize}
	
	この手法のメリットとして、次のような物が挙げられる。
	\begin{itemize}
		\item 見やすく表示するソフトウェアが多く利用しやすい
		\item 監視を行う機器に対し、監視対象の機器のIPアドレス等を登録する作業が不要である
		\item サーバのリソース監視ソリューションを利用した場合、機器ごとアプリケーションごとの監視が可能である
	\end{itemize}
	
	また、デメリットとして、次のような物が挙げられる。
	\begin{itemize}
		\item 機器ごとに、転送先を設定しなければならい
		\item 複数のソフトウェアを組み合わせるので、設定作業が負担である
	\end{itemize}

\begin{comment}
	この形だと,機器監視サーバが監視対象のIPアドレスを覚えておく必要がなくなるため,監視対象が接続されるネットワークが切り替わった場合でも,追跡が可能である.
	ところが,エージェントプログラムの導入には,技術スキルと時間が必要である.
	また,サーバーの構築・設定も必要となる.
\end{comment}

\section{株式会社SORACOMAが機能として提供している監視} %SORACOMの話を入れる。
	ネットワークの提供者がオプションとして提供している監視機能を利用する方法もある。
	ここでは、SORACOMが提供しているネットワークを利用した場合の監視を取り上げる。
	
	株式会社SORACOMとは、IoTプラットフォームとして、通信とクラウドへの接続を提供している企業である。
	通信はMVMOとして携帯電話回線網を使用しており、SORACOM用SIMを購入し、機器にSIMモジュールを取り付けることで利用できる。
	株式会社SORACOMでは、各SIMのアクティベートや通信量設定の変更、クラウドへの接続の設定等を行うためのWebインターフェースを提供しており、
	プラットフォーム利用者はこのWebインターフェースをインターネットを介して利用し、各種設定を行うことができる。
	そのWebインターフェースには、各SIMの通信量や状態を把握するためのページが存在しており、ここで各SIMがリンクアップしているかどうかを監視することもできる。

	この機能を用いた場合のメリットをあげる。
	\begin{itemize}
		\item 新たに監視サーバ等を立てる必要が無い。
		\item インターネットに接続できる環境ならば、どこからでも監視を行うことが可能である。
	\end{itemize}

	デメリットとして、下記があげられる。
	\begin{itemize}
		\item SORACOMが提供しているネットワークを利用していないと使用することができない
		\item リンクアップ、リンクダウンと通信量しか監視できない。
		\item 他のネットワークに接続された機器を監視することはできない。
	\end{itemize}

\begin{comment}
	しかし,そのネットワークを利用している場合に限り機器の状況を推測できるので,異なるネットワークに接続された機器と,専用ネットワークを用いた機器とを1箇所で監視することが難しい.
\end{comment}

\section{VPNの利用}
	VPNを利用して、既存の監視手法を使用する方法もある。
	この場合のメリットを以下に挙げる。
	\begin{itemize}
		\item 機器が利用しているネットワークによらない監視を行うことができる
	\end{itemize}
	
	この場合のデメリットを以下に挙げる。
	\begin{itemize}
		\item 機器が利用しているネットワークのアドレスが、VPNのアドレスと重複していないことが求められる。
		\item 新たに、VPNサーバをたち上げる必要がある。また、VPNの設定は繁雑であるため、負担となる。
		\item 監視サーバや機器に対し、IPアドレスを設定することや、各ソフトウェアに設定する事は、改善されない。
	\end{itemize}

\begin{comment}
	しかし、この場合でも、VPNサーバを立ち上げなければならない、監視のためのサーバを立ち上げなければならない。
	またVPNクライアントが動く機器でないといけない。
	また、IoT機器が接続するネットワークのアドレスと、VPNのネットワークアドレスがかぶっていてはならない事が求められる。
	IoT機器が接続されるネットワークが様々であることを考えると、現実的ではない。
\end{comment}


\section{既存手法のまとめ}
	既存の手法として,サーバーからの問い合わせによる監視,監視対象機器からの通知による監視,ネットワークによる監視、VPNの利用といった手法がある.
	次に、IoT機器の監視の要件をまとめる。
	\begin{enumerate}
		\item IoT機器の接続されるネットワークが、プライベートアドレスを使用していても、監視可能であること
		\item ネットワークが違っていても、一つの画面で確認できること
		\item IoTサービスの変更が不要であること
		\item 監視サーバを立てる必要がないこと
	\end{enumerate}

	次に示すのは、各要件が満たせるかどうかについての表である。
	\begin{table}[htb]
	\begin{tabular}{|l|l|l|l|l|l|} \hline
		\multicolumn{2}{|l|}{既存手法\要件番号} & 
		\multicolumn{1}{|l|}{1} &
		\multicolumn{1}{|l|}{2} &
		\multicolumn{1}{|l|}{3} &
		\multicolumn{1}{|l|}{4} \\ \hline \hline
		監視サーバからの問い合わせによる監視 & VPNと併用しない場合 & X & ◯ & ◯ & X  \\ \cline{2-6}
			& VPNと併用する場合 & ◯ & ◯ & ◯ & ☓ \\ \hline
		監視される機器からの通知による監視 & VPN と併用しない場合 & ◯ & ◯ & ◯ & X \\ \hline
		\multicolumn{2}{|l|}{ネットワーク提供者が提供する機能による監視} & ◯ & X & ◯ & ◯ \\ \hline
	\end{tabular}
	\end{table}
\begin{comment}
	しかし、サーバーからの問い合わせによる監視では、IoT機器が接続するネットワークがプライベートアドレスであった場合に利用できず、
	監視対象機器からの通知による監視では、新たに機器監視サーバをたち上げなくてはならない。
	また、ネットワークの提供者による機器の監視サービスでは、提供ネットワークを利用した機器のリンクアップ・ダウンしか監視することが出来ず、
	VPNを利用した方法では、IoT機器が接続するネットワークのアドレス帯が多様であることから、難しい。
\end{comment}
%	サーバーからの問い合わせによる監視は,現実的でなく,
	
	







