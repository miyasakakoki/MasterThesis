IoT機器の監視は、次のような要件を満たしている必要があることが分かった。
\begin{itemize}
\item IoT機器の接続されるネットワークが、プライベートアドレスを使用していても、監視可能であること\\
	IoT機器が接続されるネットワークを予め予測することは難しい。
	そのため、IoT機器にプライベートアドレスが割り振られる可能性がある。
	そのような状況でも監視することができる必要がある。
\item ネットワークが違っていても、一つの画面で確認できること\\
\item IoTサービスの変更が不要であること\\
	サーバの変更は開発者への負担となるため、サーバの変更を避けたい。
\item 監視サーバを立てる必要がないこと\\
	新たに監視サーバを立てることは、設定や構築作業が必要となるため、開発者の負担となる。
	そのため、新規に監視サーバを立てる必要が無いことが求められる。
\item 機器に異常があった場合に、運用者に通知を行う機能があること\\
	常に監視画面を見ているわけには行かないので、アラート機能等があると良い。
\end{itemize}

そこで、従来からある手法にて対応できるか、考えた。
\section{サーバからの問い合わせによる監視}
	従来から機器監視に用いられてきた手法として,定期的に機器監視サーバから対象機器に状態を問い合わせる手法がある.
	機器監視サーバから,監視対象機器上のエージェントプログラムに,現在の状態を問い合わせる事で機器の監視を実現している.
	\medskip

	代表的なものとして、次のような物がある。
	\begin{itemize}
		\item Ping\\
			Pingとは、ICMPメッセージをやり取りするプログラムである。
			ICMPとは、InternetControlManagementProtocolの事で、IPネットワークにおいて、IPパケットの不到達等を通知する為の取り決めである。
			ICMPには、ICMP Echo Request Echo Replyが定義されており、ICMP Echo Requestを受け取った機器は、ICMP Echo Replyを返さなくてはならない。
			Pingは、ICMP Echo Requestを指定したIPアドレスに送信し、EchoReplyの有無から、機器の存在や生存を確認する事ができる。
		\item SNMPによる問い合わせ\\
			SNMPとは、Simple Network Management Protocolの事で、ネットワークにおいて、機器の状態を確認するために存在する。
			SNMPでは、サーバから機器の状態を問い合わせるメッセージが定義されており、それを用いてサーバーから機器の状態を確認する事ができる。
	\end{itemize}
	
	この手法のメリットを以下にまとめる。
	\begin{itemize}
		\item 手軽に機器の状態やネットワークの状態を確認することができる
		\item 見やすく表示する事ができるソフトウェアが豊富に存在する
		\item コンピューターで有ればほぼ全ての機器が対応している
	\end{itemize}

	また、デメリットとして、次のような物がある。
	\begin{itemize}
		\item IPアドレスで機器を識別することになるので、監視サーバが接続されたネットワークとは違うネットワークに接続された機器を監視することは困難である。
			間にあるネットワーク機器の設定を全て変更しなくてはならないため、大変である。
		\item 機器は常に待ち受けていなくてはならないため、省電力を目的としたスリープをすることが難しい
		\item IPアドレスで機器を識別することになるので、NAPTの内側にある場合、機器を識別することが出来ない
		\item 攻撃の足がかりとして使われる場合もあるので、使用できない場合がある
	\end{itemize}

	IoT機器の監視は、機器がNAPTの内側にある場合が多く、ユーザーに近い場所に設置されることが多いので、この方法を用いるためには、
	機器がグローバルIPアドレスを持っていること、間のネットワークにて監視用パケットがブロックされない様ネットワーク機器を設定すること、
	が必要である。

	この形では,機器監視サーバは,監視対象機器のIPアドレスを覚えておかねばならない.
	そのため,監視対象が接続されるネットワークが切り替わった場合,監視サーバ上に記録されたIPアドレスを変更しなくては監視できない.
	監視対象が,多量且つ移動するIoT機器の監視において,頻繁に監視サーバ上に記録されたIPアドレスを書き換えるのは大変である.
	また,プライベートアドレスの利用時には,サーバーからIoT機器への到達性が失われる可能性がある.

\section{監視対象機器からの通知による監視}
	機器監視手法として,定期的に監視対象機器から機器監視サーバーへ状態を送信するという手法がある.
	監視対象機器上のエージェントプログラムが,指定した時間毎に機器監視サーバーへ状態を送信することで,機器の監視を実現している.
	\medskip
	
	代表的なものとして、次のようなものがある。
	\begin{itemize}
		\item SNMP Trapによる取得\\
			SNMPTrapとは先に述べたSNMPのTrapである。
		\item サーバーのリソース監視ソリューション\\
			具体的なものとして、Teregraf+Elasticsearch+Kibanaを使用したサーバー監視ソリューションがあげられる。
	\end{itemize}
	
	この手法を取るメリットは、
	1.機器から状態を送信することで、NAPTの内側に機器が設置されていても問題ない事
	2.省電力を目的としたスリープをすることができること
	がある。

	また、デメリットとして、
	1.SNMPを利用している場合は、ブロックされてしまうことがあること
	2.監視サーバの設定が繁雑であること
	がある。

	この形だと,機器監視サーバが監視対象のIPアドレスを覚えておく必要がなくなるため,監視対象が接続されるネットワークが切り替わった場合でも,追跡が可能である.
	ところが,エージェントプログラムの導入には,技術スキルと時間が必要である.
	また,サーバーの構築・設定も必要となる.

\section{ネットワークの提供者による機器の監視サービス} %SORACOMの話を入れる。
	ネットワーク機器から機器の監視を行うこともできる。

	代表的なものとして、株式会社SORACOMが行っているサービスが挙げられる。

	株式会社SORACOMでは、IoTプラットフォームとして、通信とクラウドを提供するサービスを行っている。
	通信はMVMOとして携帯電話網を使用しており、SORACOMAir用SIMを購入し、機器にSIMモジュールを取り付けることで利用できる。

	そもそも監視対象機器がつながる為のネットワークを作ってしまおうという動きもある.
	既設の携帯電話網を利用して,IoT機器用ネットワークサービスが展開されている.
	これは,IoT機器向けのデータ通信を提供するサービスで,Web画面から通信量や接続状況を確認することができる.
	携帯電話網を利用しているので,接続状況から機器の状態を推測できる.

	しかし,そのネットワークを利用している場合に限り機器の状況を推測できるので,異なるネットワークに接続された機器と,専用ネットワークを用いた機器とを1箇所で監視することが難しい.

\section{VPNの利用}
	VPNを利用して、既存の監視ソリューションを使用する方法もある。
	しかし、この場合でも、VPNサーバを立ち上げなければならない、監視のためのサーバを立ち上げなければならない。
	またVPNクライアントが動く機器でないといけない。
	また、IoT機器が接続するネットワークのアドレスと、VPNのネットワークアドレスがかぶっていてはならない事が求められる。
	IoT機器が接続されるネットワークが様々であることを考えると、現実的ではない。

\section{既存手法のまとめ}
	既存の手法として,サーバーからの問い合わせによる監視,監視対象機器からの通知による監視,ネットワークによる監視、VPNの利用といった手法がある.
	しかし、サーバーからの問い合わせによる監視では、IoT機器が接続するネットワークがプライベートアドレスであった場合に利用できず、
	監視対象機器からの通知による監視では、新たに機器監視サーバをたち上げなくてはならない。
	また、ネットワークの提供者による機器の監視サービスでは、提供ネットワークを利用した機器のリンクアップ・ダウンしか監視することが出来ず、
	VPNを利用した方法では、IoT機器が接続するネットワークのアドレス帯が多様であることから、難しい。
%	サーバーからの問い合わせによる監視は,現実的でなく,
	
	







