序論で述べたとおり、本研究で解決する問題は以下の3つである。
\begin{itemize}
	\item 設置したものが正常に稼働し始めたかどうか確認するのが面倒\\
	設置者が、デバイスの操作を知っている必要が有る。\\
	また、ディスプレイ等をつけないことが多いので、別途確認する手段(ディスプレイとキーボードを持参等)を用意する必要がある。
	\item 設置後、正常に稼働しているのか確認するのが面倒\\
	NAPTの内側に設置されている事が多いので、Pingやsnmpでは確認できない。\\
	また、ネットワークの断絶等があった場合、稼働状況を確認できない。
	\item いつ稼働していていつ稼働していなかったのか管理するのが大変\\
	いつ稼働していていつ稼働していなかったのかがわからないと、データを正確に分析する事が出来ない。
\end{itemize}

以下に状態監視のシステムを導入?しない場合の解決策を上げる

\section{デバイス設置箇所に行って、直接確認する}
直接設置箇所に行き、確認を行うという方法である。
この場合、ディスプレイとキーボードを持参するなどする必要がある。
そして、デバイスを良く知る者を行かせる必要もある。
設置台数が多いことや、設置個所が離れていることから、あまり現実的ではない。

\section{ICMP Pingを活用する}
ICMPとは、InternetControlMessageProtocolの略であり、IPパケットの送り元から送り先への間で起きた問題を通知する役割を持つ。
ICMPには、ICMP echo requestと、ICMP echo repryが定義されており、ICMP echo requestを受け取った機器は、ICMP echo repryを返送しなければならない。
PingはICMP echoを送信する為のプログラムで、IP網のトラブルの発見の他、特定のIPアドレスを持つ機器が稼働しているかどうか確認するためにも使われている。
Pingを使用して、IoTデバイスの稼働確認を行うというのがこの解決策である。
しかし、ICMPパケットは、セキュリティの都合上、ネットワーク機器で転送しないよう設定されている場合が多い。
また、一般的なネットワークでは、インターネットとの接続点にてネットワークを分離している事がある。
その場合、IoTデバイスのIPアドレスは、IoTデバイスの所属するネットワークでのみ通用するアドレスとなるため、外部からICMPパケットを送ることは出来ない。

%なんかいまいち説明になっていないのでコメントアウト
\begin{comment}
NAPTとは、NetworkAddressPortTranslationの略で、外部のネットワークと内部のネットワークを分離する機能を持つ。
そのため、デバイス管理者から送信された
具体的には、内側から外側へデータを送信する際、送信元IPアドレスとポート番号を機器のIPアドレスとポート番号に変換・記憶し、
外部から内部へデータの送信があった場合、記憶していたIPアドレスとポート番号から、送信先IPアドレスとポートを書き換え、内部へ転送する
そのため、内側から始る通信は問題ないが、外側から始まる通信はブロックしてしまう。
\end{comment}

\section{SNMPを利用する}
SNMPとは、SimpleNetworkManagementProtocolの略で、ネットワークに接続された機器を監視するために作られている。
SNMPでは、

しかし、これらの手法では、解決に至っていない。
そこで、通常(?)は、次のような方法で解決を図っている。
\section{Zabbixを使用する}
\section{Fluentd Elasticksearch Kibanaを利用する}
\section{Telegraf Influxdb Grafanaを利用する}
しかし、これらの解決策は大変だ。


\begin{comment}
それぞれにおける既存の解決策と利点と欠点を以下に述べる
\section{設置したものが正常に稼働し始めたかどうか確認するのが面倒}
\subsection{デバイスにディスプレイとキーボードをつける}
\subsubsection{個々のデバイスごとにつける場合}
デバイス1台あたりのコストがかさむ
勝手に操作されることがある(セキュリティーの問題)
\subsubsection{設置者が持参しつける場合}
デバイスにディスプレイを接続できる必要がある。
より安価なデバイスだと、ディスプレイを接続できない場合が多い。
勝手にディスプレイ等をつけ、操作されることがある(セキュリティーの問題)
\subsection{PCを持参し、接続する場合}
PCと接続できるインターフェースを用意する必要がある。
勝手にPCを繋がれるかもしれない。

上記2つは勝手に操作される可能性がある。

\subsection{遠隔からログインし、状態を確認する場合}
間にNAPTが挟まっている場合があるのでできない事がある。

また、上記3つとも共通だが、デバイスに対して知識のある人物が確認しなければならない。
設置場所が離れている場合は、それ相応のコストがかかる。
\subsection{デバイスにLED等の簡素なディスプレイをつける}
最も安価だが、デバイスにLEDをつけるのが、(そこまででも無いけど)めんどい。
また、詳しいことはわからない。



\section{設置後、正常に稼働しているのか確認するのが面倒}
\subsection{デバイスを直接現地に行って確認する}
設置場所が離れている場合が多いので、コストがかかる。
\subsection{遠隔から確認する}
\subsubsection{ICMP Pingを用いる}
ICMPとは:
	

この解決策の利点:
欠点としては、NAPTが障害となり、遠隔から確認することは無理。
NAPTとは:
\subsubsection{SNMPを用いる}
SNMPとは:

NAPTが障害となり、遠隔から確認することは無理。
\subsubsection{Fluentd Elastickserch Kibanaを使う。}
Fluentdとは:
ElastickSerchとは:
Kibanaとは:

Fluentdが入るようなデバイスでないといけない。
また、サービス毎にグラフ描画の設定をしなければならない。
\subsubsection{Zabbix}
(調査中?)
\subsubsection{Telegraf + Influxdb + Grafana}
Telegrafとは
Influxdbとは
Grafanaとは
Telegrafが入るようなデバイスでないといけない。
また、サービス毎にグラフ描画の設定をしなければならない。
\section{いつ稼働していて稼働していなかったのか管理するのが大変}
\subsection{ログファイルを回収して確認}
現地に行くのめんどい
\subsection{Fluentd Elasticksearch Kibanaを使う}
Fluentdが入るようなデバイスでないといけない。
また、サービス毎にグラフ描画の設定をしなければならない。

\subsection{Telegraf + Influxdb + Grafanaを使う}
Telegrafが入るようなデバイスでないといけない。
また、サービス毎にグラフ描画の設定をしなければならない。

\end{comment}

