序論で述べたとおり、本研究で解決する問題は以下の3つである。
\begin{itemize}
	\item 設置したものが正常に稼働し始めたかどうか確認するのが面倒\\
	設置者が、デバイスの操作を知っている必要が有る。\\
	また、ディスプレイ等をつけないことが多いので、別途確認する手段(ディスプレイとキーボードを持参等)を用意する必要がある。
	\item 設置後、正常に稼働しているのか確認するのが面倒\\
	NAPTの内側に設置されている事が多いので、Pingやsnmpでは確認できない。\\
	また、ネットワークの断絶等があった場合、稼働状況を確認できない。
	\item いつ稼働していていつ稼働していなかったのか管理するのが大変\\
	いつ稼働していていつ稼働していなかったのかがわからないと、データを正確に分析する事が出来ない。
\end{itemize}

それぞれにおける既存の解決策と利点と欠点を以下に述べる
\section{設置したものが正常に稼働し始めたかどうか確認するのが面倒}
\subsection{デバイスにディスプレイとキーボードをつける}
\subsubsection{個々のデバイスごとにつける場合}
デバイス1台あたりのコストがかさむ
勝手に操作されることがある(セキュリティーの問題)
\subsubsection{設置者が持参しつける場合}
デバイスにディスプレイを接続できる必要がある。
より安価なデバイスだと、ディスプレイを接続できない場合が多い。
勝手にディスプレイ等をつけ、操作されることがある(セキュリティーの問題)
\subsection{PCを持参し、接続する場合}
PCと接続できるインターフェースを用意する必要がある。
勝手にPCを繋がれるかもしれない。

上記2つは勝手に操作される可能性がある。

\subsection{遠隔からログインし、状態を確認する場合}
間にNAPTが挟まっている場合があるのでできない事がある。

また、上記3つとも共通だが、デバイスに対して知識のある人物が確認しなければならない。
設置場所が離れている場合は、それ相応のコストがかかる。
\subsection{デバイスにLED等の簡素なディスプレイをつける}
最も安価だが、デバイスにLEDをつけるのが、(そこまででも無いけど)めんどい。
また、詳しいことはわからない。


\section{設置後、正常に稼働しているのか確認するのが面倒}
\subsection{デバイスを直接現地に行って確認する}
設置場所が離れている場合が多いので、コストがかかる。
\subsection{遠隔から確認する}
\subsubsection{ICMP Pingを用いる}
NAPTが障害となり、遠隔から確認することは無理。
\subsubsection{SNMPを用いる}
NAPTが障害となり、遠隔から確認することは無理。
\subsubsection{Fluentd Elastickserch Kibanaを使う。}
Fluentdが入るようなデバイスでないといけない。
また、サービス毎にグラフ描画の設定をしなければならない。
\subsubsection{Zabbix}
(調査中?)
\subsubsection{Telegraf + Influxdb + Grafana}
Telegrafが入るようなデバイスでないといけない。
また、サービス毎にグラフ描画の設定をしなければならない。
\section{いつ稼働していて稼働していなかったのか管理するのが大変}
\subsection{ログファイルを回収して確認}
現地に行くのめんどい
\subsection{Fluentd Elasticksearch Kibanaを使う}
Fluentdが入るようなデバイスでないといけない。
また、サービス毎にグラフ描画の設定をしなければならない。

\subsection{Telegraf + Influxdb + Grafanaを使う}
Telegrafが入るようなデバイスでないといけない。
また、サービス毎にグラフ描画の設定をしなければならない。



