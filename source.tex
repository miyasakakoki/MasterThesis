\section*{ソースコード概要}
作成したソースコードはgithubにもアップロードされている.
\url{https://github.com/miyasakakoki/devicemonit/tree/feature}

作成したプログラムは大きく分けて3つある.
プログラム名の前に付いているのは、ファイル名である。
\begin{itemize}
\item エージェントプログラム\\
	\begin{description}
		\item[agent/agent.sh] エージェントプログラム
	\end{description}
\item エージェントプログラム用インターフェース\\
	\begin{description}
		\item[api/api.py] エージェントプログラム用インターフェース本体
	\end{description}
\item 監視アプリケーション\\
	\begin{description}
		\item[ui/devicemonit.py] 監視アプリケーション本体
		\item[ui/schema.sql] 機器情報データベース作成用スクリプト
		\item[ui/templates/outerlayout.html] Webページ全体のレイアウト用テンプレート
		\item[ui/templates/layout.html] ログインページを除くWebページ用のテンプレート
		\item[ui/templates/login.html] ログインページ
		\item[ui/templates/dashboard.html] 機器情報一覧ページ
		\item[ui/templates/devicelog.html] 過去の機器状態一覧ページ
	\end{description}
\end{itemize}


\section*{エージェントプログラム(agent/agent.sh)}
このプログラムは、IoT機器にインストールされる。シェルスクリプトとして書かれている
実行の際、引数として、機器IDを指定する。
\lstinputlisting[caption=エージェントプログラム,label=agnt]{src/agent/agent.sh}

\section*{エージェントプログラム用インターフェース(api/api.py)}
このプログラムは、サーバ上で動作する。
プログラムを作動させるには、Python3,gunicorn等のWSGIサーバがインストールされている必要がある。
また、ライブラリとしてFalcon,InfluxDBClientがインストールされている必要がある。
gunicornを用いた際の起動の仕方は次のようになっている。
%起動方法

以下にソースコードをあげる。
\lstinputlisting[caption=エージェントプログラム用インターフェース,label=interface]{src/api/api.py}

\section*{監視アプリケーション(ui/)}
監視アプリケーションは、Flaskを用いたサーバーサイドプログラムと、Webブラウザ上で動作するJavascriptが埋め込まれたWebページに分けることができる。
Webページは、Flaskに内蔵されているレンダリングエンジンによって、複数のテンプレートファイルから生成される。
アプリケーションの起動には、Python3がインストールされている必要がある。
また、Flask,InfluxDBClient,sqlite3がインストールされている必要がある。
サーバマシン上で次の操作を行うことで起動することができる。
%起動方法

\subsection*{監視アプリケーション本体(ui/devicemonit.py)}
このプログラムは、サーバーサイドで動作するプログラムである。
データベースの初期化の為に機器情報データベース作成用スクリプトを読み込み、実行している。
\lstinputlisting[caption=監視アプリケーション本体,label=uiapp]{src/ui/devmonit.py}

\subsection*{機器情報データベース作成用スクリプト(ui/schema.sql)}
このスクリプトは、データベースの初期化の際に、アプリケーション本体によって読み込まれる。
\lstinputlisting[caption=機器情報データベース作成用スクリプト,label=schema]{src/ui/schema.sql}


\subsection*{テンプレートファイル郡}
これらは、利用者がブラウザからアプリケーションにアクセスされた際に、Flaskのレンダリングエンジン(jinja)によって読み込まれ、処理される。
Javascript、HTML/CSSと、jinja用スクリプトによって書かれている。
\begin{description}
\item[ui/templates/outerlayout.html] Webページ全体のレイアウト用テンプレート\\
	このテンプレートファイルは、アプリケーションが出力する全てのWebページに適応される。\\
	\lstinputlisting[caption=全体のレイアウト用テンプレート,label=outerl]{src/ui/templates/outerlayout.html}
\item[ui/templates/layout.html] ログインページを除くWebページ用のテンプレート\\
	このテンプレートファイルは、ログインページ以外のWebページに適応される。\\
	\lstinputlisting[caption=レイアウト用テンプレート,label=layout]{src/ui/templates/layout.html}
\item[ui/templates/login.html] ログインページ\\
	\lstinputlisting[caption=ログインページ,label=login]{src/ui/templates/login.html}
\item[ui/templates/dashboard.html] 機器情報一覧ページ\\
	\lstinputlisting[caption=機器情報一覧ページ,label=dashboard]{src/ui/templates/dashboard.html}
\item[ui/templates/devicelog.html] 過去の機器状態一覧ページ\\
	\lstinputlisting[caption=過去の機器状態一覧ページ,label=devicelog.html]{src/ui/templates/devicelog.html}
\end{description}
