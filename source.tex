\subsection{ソースコード概要}
作成したソースコードはgithubにもアップロードされている。
\url{https://github.com/miyasakakoki/devicemonit/tree/feature}

作成したプログラムは大きく分けて3つある。
\begin{itemize}
\item エージェントプログラム 1点\\
	\begin{description}
		\item[agent/agent.sh] エージェントプログラム
	\end{description}
\item エージェントプログラム用インターフェース 1点\\
	\begin{description}
		\item[api/api.py] エージェントプログラム用インターフェース本体
	\end{description}
\item 監視アプリケーション 7点\\
	\begin{description}
		\item[ui/devicemonit.py] 監視アプリケーション本体
		\item[ui/schema.sql] 機器情報データベース作成用スクリプト
		\item[ui/template/outerlayout.html] Webページ全体のレイアウト用テンプレート
		\item[ui/template/layout.html] ログインページを除くWebページ用のテンプレート
		\item[ui/template/login.html] ログインページ
		\item[ui/template/dashboard.html] 機器情報一覧ページ
		\item[ui/template/devicelog.html] 過去の機器状態一覧ページ
	\end{description}
\end{itemize}

\subsection{エージェントプログラム(agent/agent.sh)}

\subsection{エージェントプログラム用インターフェース(api/api.py)}

\subsection{監視アプリケーション本体(ui/devicemonit.py)}

\subsection{機器情報データベース作成用スクリプト(ui/)
\item[ui/template/outerlayout.html] Webページ全体のレイアウト用テンプレート
\item[ui/template/layout.html] ログインページを除くWebページ用のテンプレート
\item[ui/template/login.html] ログインページ
\item[ui/template/dashboard.html] 機器情報一覧ページ
\item[ui/template/devicelog.html] 過去の機器状態一覧ページ
