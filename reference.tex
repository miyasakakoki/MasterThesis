
\begin{thebibliography}{99}
\addcontentsline{toc}{chapter}{\bibname}
\bibitem{エコパ} eCoPA \url{http://ecopa.in/}

\bibitem{情報通信白書} 平成27年版 情報通信白書(総務省) 「IoTの実現に向けたアプローチと我が国ICT産業の方向性」より \url{http://www.soumu.go.jp/johotsusintokei/whitepaper/ja/h27/html/nc254150.html}
\bibitem{IoT流行語} 6割が「IoTは流行語」--エスキュービズム調査 (ZDBet Japan) \url{http://japan.zdnet.com/article/35093272/}
\bibitem{ガートナー調査} 先進テクノロジのハイプ・サイクル2011年 Gartner \url{https://www.gartner.co.jp/press/html/pr20110907-01.html}


\bibitem{SORACOM概要} 株式会社SORACOM「SORACOMの概要」 \url{https://soracom.jp/overview/} (2017年2月7日 閲覧)
\bibitem{SORACOM記事1} TechCrunch Japan「【詳報】ソラコムがベールを脱いだ、月額300円からのIoT向けMVNOサービスの狙いとは?」 \url{http://jp.techcrunch.com/2015/09/30/soracom-launches-mvno-service-for-iot/} (2017年2月7日 閲覧)
\bibitem{SORACOM記事2} 株式会社沖縄アイオー 金城辰一郎 「ソラコムによるIoTサービス内容とは?非エンジニアがその革新的な魅力と導入事例をわかりやすく徹底解説」 \url{http://okinawa.io/blog/tech/soracom_iot} (2017年2月7日 閲覧)
\bibitem{SORACOM記事3} IoTNews.jp 小泉耕ニ 「【前編】SORACOMが発表した新サービスは、なにがすごいのか? SORACOM Connected」 \url{https://iotnews.jp/archives/12037} (2017年2月7日 閲覧)
\bibitem{SORACOM記事4} THE BRIDGE Takeshi Hirano 「SORACOMの凄さは第三者が「SIM」を自由に発行・運用できることーーIoT向けモバイル通信PF、ソラコムが提供開始」 \url{http://thebridge.jp/2015/09/soracom} (2017年2月7日 閲覧)
\end{thebibliography}
