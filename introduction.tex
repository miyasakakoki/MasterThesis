本章では、論文の構成のと、研究の背景及び現状の課題について記述し, 本研究の目的について述べる.

\section{本論文の構成}
本論文では、第一章にて研究の背景及び現状の課題、本研究の目的について述べる。
第二章では、既存の解決策とその課題を分析し、
第三章にて、サービスの開発を提案し、そのための要求の分析を行い、
第四章では、サービスの構成と機能、ユーザーの動きについて述べる。
第五章では、検証として行った実験と考察、
第六章では、成果と今後の課題について述べる。

\section{研究の背景}
近年, 半導体技術の進歩により, コンピューターの小型化・低価格化が進んでいる. 
また, 家庭へのインターネットの普及により, 全ての物がインターネットに接続し相互に情報を交換し合い様々な自動化を実現するIoTが注目されている.\\

それを意識して、IoTデバイスとして簡単に使用することができるコンピューターが登場している。
\begin{itemize}
\item RaspberryPi\\
	Linux OSが動く3000円前後のコンピューターとして2012年に発売される。
	教育用の小型コンピュータとして開発された経緯があるが、他のハードウェアとの接続を考えたGPIOや、ネットワークへの接続するためのEthernetインターフェースがあることから、IoT開発にて頻繁に使われている。
	また、より小型化を図ったRaspberryPiZeroや、組み込み機器への搭載を考えたComputeModule等、バリエーションが豊富である。
\item Intel Edison\\
	当初より、IoTを意識して作られたコンピュータ。2014年に発売された。
	RaspberryPiとは違い、IoT向けに作られているので、ディスプレイポートや、音声の出力ポートは存在しない。
	また、SDカード大のサイズと、とても小型である。
	Wifi,Bluetoothのインターフェースの他、USBインターフェース、GPIOが付いている。
\end{itemize}
また、よりIoTに特化したものとしてWioNodeという物がある。
\begin{itemize}
\item WioNode\\
	SeedStudioが開発したIoTデバイス。
	RaspberryPiにつなげて使うIoTデバイス「GrovePi」の各モジュールが、使用できる。
	また、Wifi経由でスマートフォンをつなぎ設定できるのも特徴の一つである。
	Wifi付きのマイコンボードとしても利用できる。
\end{itemize}

以下のようにそれらを使ったIoTサービスの開発も盛んに行われている。
\begin{itemize}
\item 太陽光発電の発電量の監視
	ある株式会社R社様では、太陽光発電所の発電量を監視し、配信するためにIoTを使用している。
	太陽光発電所では、太陽光パネルで発電した電力をパワーコンディショナに繋ぎ、そこから送電線へ送電している。
	R社様では、そのパワーコンディショナにR社製機器を繋ぎ、インターネットを経由して情報を送るサービスを展開している。
\item 農業の自動化
	
\item パスの運行情報の掲示
	
\end{itemize}

ここで、IoTサービスとは、IoTによる自動化を提供する物のうち、デバイスから得た情報を蓄積・分析し、結果を元に、表示等の動作を行うものと定義する。
IoTサービスの構成としては、複数のデバイスから1つのコンピューターへ情報を送り、その上で上で蓄積・解析し、結果を表示する等の動作を行っているものが多い。

\section{問題}
このように、IoTデバイスの価格が下がることで、IoTサービスの開発にかかるコストが低減され、開発への垣根が下がる一方で、
サービスの運用において、次のような問題がある。
\begin{itemize}
	\item 数が多くて管理しきれない問題
	\begin{itemize}
		\item 設置前の設定において、どのデバイスをどこに設置すれば良いのかわからなくなる -> ラベリングにて解決
		\item 設置後、どのデバイスがどこに設置されたのかわからなくなる -> 帳簿をつけることで解決
		\item 設定の際に、個別の設定をしなければならないのが面倒\\
			具体的には、デバイスに振るID等。\\
			ラベリングと整合性が取れていなければならない。
	\end{itemize}
	\item 稼働状況の監視が面倒な問題
	\begin{itemize}
		\item 設置したものが正常に稼働し始めたかどうか確認するのが面倒\\
			設置者が、デバイスの操作を知っている必要が有る。\\
			また、ディスプレイ等をつけないことが多いので、別途確認する手段(ディスプレイとキーボードを持参等)を用意する必要がある。
		\item 設置後、正常に稼働しているのか確認するのが面倒\\
			NAPTの内側に設置されている事が多いので、Pingやsnmpでは確認できない。\\
			また、ネットワークの断絶等があった場合、稼働状況を確認できない。
		\item いつ稼働していていつ稼働していなかったのか管理するのが大変\\
			いつ稼働していていつ稼働していなかったのかがわからないと、データを正確に分析する事が出来ない。
	\end{itemize}
\end{itemize}

稼働状況の監視については、IoTサービスで行うことがある程度可能だが、サービス自体に手を加える必要があるため、開発のコストが高くなる。

その中で、私は、IoTデバイスの状態監視に着目した。

\section{研究の目的}
そこで、IoTサービスとは独立したIoTデバイスの監視サービスを開発することにより、これらの問題を解決できるのではないかと考えた。
本研究では, IoTデバイスの監視サービスを開発することで、IoTデバイスの状態監視を簡単化することを目的とする。

