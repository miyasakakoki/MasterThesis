\begin{comment}
IoTサービスというものがある。
IoTサービスはこんな構造をしている。
だけど、IoTサービスの開発・運用は大変
・開発者が開発するのが大変
・運用者が維持することが大変

なので、監視が必要です。

本論文では、〇〇と△△といった方法で、監視システムを開発することにより、
IoTサービスの開発者と運用者の負担を軽減する事を目指す。
\end{comment}



\section{研究の背景}
近年、IoTが注目を集めている。
IoTとは、様々なモノがインターネットにつながり、相互に情報をやり取りすることで、利便性の向上を図る概念である。
近年の半導体技術の進歩により,コンピュータが安価・小型になったこと,インターネットへの通信が様々な場所で安価に行えるようになったことにより,注目が集まっている.
IoTの具体的な例としては、建設重機の盗難防止、プリンタのトナー発注自動化、体温や脈拍等の収集・可視化等が挙げられる。
\medskip

IoTサービスとは、IoTによる利便性を顧客に提供するサービスの事で、より生活に身近なサービスの登場が期待されている。
上記の例では、プリンタのトナー発注自動化がIoTサービスにあたる。
IoTサービスは、IoT機器とサーバがインターネットを介して通信し合うことで成り立っている。
IoT機器は,モノにコンピュータが取り付けられた物で,周囲の状況を検知,または,周囲へ働きかける機能を持つ.
サーバーは,IoT機器からの情報を蓄積・分析し,IoT機器へ指示を送るか,ユーザーへ分析結果を表示する機能を持つ.
これらIoT機器とサーバーが連携することで,IoTサービスは利便性をユーザーへ提供している.

\section{IoTサービスの提供者の課題}
IoTサービスを円滑に提供するには,IoT機器とサーバーの連携を正常に維持しなければならない.
しかし、サーバーの監視は既存手法で解決できるが、IoT機器の監視には、次のような問題があり、問題に対処する事が提供者の負担となっている。
\begin{itemize}
\item IoT機器が接続されるネットワークの構成は様々であり、接続されるネットワークの変更も出来ないことが多い。
\item IoTサービスが提供する機能と、ネットワークや機器の状態の監視は別機能である為、IoTサービスに組み込む事は困難である。
\item IoT機器は安価であるため、大量に使用される。そのため、設定が困難である。
\end{itemize}


\section{研究の目的}%(何を提案するのか)
そこで、前述の課題を解決するために、新規に監視サービスを開発し、IoT機器から通知を送ることにより、IoT機器の監視における困難を軽減することを目的とする。
%そこで,我々は,IoT機器からの通知に基づいた機器監視サービスを提案する.
IoT機器が自身の過去の動作状態や通信状態を記録することで,設置されるネットワークに関係なく状態監視をすることを可能にする.
また,サービスを機器監視に特化させ独立させることで,IoTサービスに変更を与えること無く,機器を監視することを可能にする.
この仕組みを用いることで,IoT機器が設置されるネットワークに関係無く状態を監視することや,IoT機器監視の設定の簡略化や、IoTサービスに組み込む負担の軽減を行う。
本研究では,IoT機器からの通知による機器の設置環境によらない機器の監視を行うことにより,IoTサービスの維持を容易にするシステムの開発に取り組む.

\section{本論文の構成}%(論文の構成)
本論文では,IoT機器の監視困難の問題を取り上げ,その問題解決のための監視サービスを開発し,効果を報告する.
第2章では,IoTサービスの維持に関する背景と,IoT機器の監視に関する問題を述べる.
第3章では,第2章で述べた問題を分析し,IoT機器監視サービスの機能要件について述べる.
第4章では,IoT機器監視サービスの実装の詳細について述べる.
第5章では,実験によりIoT機器監視サービスがもたらす効果を検証し,考察を述べる.
第6章では,本研究に関する評価について述べる.
第7章では,本研究を通して得られた知見や今後の課題について述べる.


\begin{comment}
\section{書き置き}
%(IoTとは何なのか)
近年,IoTが注目を集めている.
IoTとは、様々なモノがインターネットにつながり相互に情報をやり取りすることで、利便性の向上を図る概念である。
近年の半導体技術の進歩により,コンピュータが安価・小型になったこと,インターネットへの通信が様々な場所で安価に行えるようになったことにより,注目が集まっている.
\medskip

%(IoTサービスとは何なのか,何故期待されているのかを説明)
IoTサービスとは、IoTによる利便性をユーザーに提供するサービスの事で,より生活に身近なサービスの登場が期待されている.
IoTサービスは,IoT機器とサーバーがインターネットを介して通信し合うことで,成り立っている.
IoT機器は,モノにコンピュータが取り付けられた物で,周囲の状況を検知,または,周囲へ働きかける機能を持つ.
サーバーは,IoT機器からの情報を蓄積・分析し,IoT機器へ指示を送るか,ユーザーへ分析結果を表示する機能を持つ.
これらIoT機器とサーバーが連携することで,IoTサービスは利便性をユーザーへ提供している.
\medskip

%(IoTサービスの提供にはどのような事が必要となるのか)
IoTサービスを円滑に提供するには,IoT機器とサーバーの連携を正常に維持しなければならない.
そのため,IoT機器の動作状態や通信状態の監視が重要となる.
%(IoT機器の動作状態や通信状態の監視が,何故困難なのか)
数も多く,さまざまなネットワークを介して接続されるIoT機器の監視は困難な問題である.
IoT機器が設置される様々なネットワークの構成を把握することは,IoT機器が多量であることを考えると現実的ではない.
多量のIoT機器を個々に識別し,異常を検知することも難しい.
%(それらを解決する為に,何が求められるのか)
そのため,設置されるネットワークに関係なく状態が監視できることが求められる.
また,IoT機器の状態を一覧して確認できることや,IoT機器の過去の動作状態や通信状態を確認できることが必要である.
\medskip

%(何を提案するのか)
そこで,我々は,IoT機器からの通知に基づいた機器監視サービスを提案する.
IoT機器が自身の過去の動作状態や通信状態を記録することで,設置されるネットワークに関係なく状態監視をすることを可能にする.
また,サービスを機器監視に特化させ独立させることで,IoTサービスに変更を与えること無く,機器を監視することを可能にする.
この仕組みを用いることで,IoT機器が設置されるネットワークに関係無く状態を監視することや,IoT機器の過去の状態や通信状態を確認することを容易にする.
本研究では,IoT機器からの通知による機器の設置環境によらない機器の監視を行うことにより,IoTサービスの維持を容易にするシステムの開発に取り組む.
\medskip

%(論文の構成)
本論文では,IoT機器の監視困難の問題を取り上げ,その問題解決のための監視サービスを開発し,効果を報告する.
第2章では,IoTサービスの維持に関する背景と,IoT機器の監視に関する問題を述べる.
第3章では,第2章で述べた問題を分析し,IoT機器監視サービスの機能要件について述べる.
第4章では,IoT機器監視サービスの実装の詳細について述べる.
第5章では,実験によりIoT機器監視サービスがもたらす効果を検証し,考察を述べる.
第6章では,本研究に関する評価について述べる.
第7章では,本研究を通して得られた知見や今後の課題について述べる.





2.SORACOM
ソラコムでも関しをしています。
SORACOMとは、
ー>技術的な構造
考察()


\end{comment}
