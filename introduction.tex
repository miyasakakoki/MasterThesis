本章では、研究の背景及び現状の課題について記述し, 本研究の目的について述べる.

\section{研究の背景}
近年, 半導体技術の進歩により, コンピューターの小型化・低価格化が進んでいる. 
また, 家庭へのインターネットの普及により, 全ての物がインターネットに接続し相互に情報を交換し合い様々な自動化を実現するIoTが注目されている.\\

%RaspberryPiについて
%Linuxが動く3000円のコンピューターとして2012年に発売される。
%以降、価格を抑えながら性能の向上を図ってきた。
%raspberrypi Zeroの発表->低価格かつ小型になる

%IntelEdison
%IoTを意識した製品。RaspberryPiに比べると、値段は高いが、とても小型。Wifi,Bluetoothが使える。

%最近ではWioNodeという物も出てきている。
%ESP-WROOM-02 の登場
%Ardiuno
%beaglebone



このように、IoTサービスの開発が盛んに行われている。
%IoTのサービスがいくつか開発されている
%例えば、
%バスの運行情報の掲示
%農業の自動化
%太陽光発電の発電量監視


\section{問題}%問題
このように、IoTデバイスの価格が下がることで、IoTサービスの開発にかかるコストが低減され、開発への垣根が下がる一方で、
サービスの運用において、次のような問題がある。
・数が多くて管理しきれない問題
 1.設置前の設定において、どのデバイスをどこに設置すれば良いのかわからなくなる -> ラベリングにて解決
 2.設置後、どのデバイスがどこに設置されたのかわからなくなる -> 帳簿をつけることで解決
 3.設定の際に、個別の設定をしなければならないのが面倒
  具体的には、デバイスに振るID等。
  ラベリングと整合性が取れていなければならない。
・稼働状況の監視が面倒な問題
 1.設置したものが正常に稼働し始めたかどうか確認するのが面倒
  設置者が、デバイスの操作を知っている必要が有る。
  また、ディスプレイ等をつけないことが多いので、別途確認する手段(
		ディスプレイとキーボードを持参等)
	を用意する必要がある。
 2.設置後、正常に稼働しているのか確認するのが面倒
  NAPTの内側に設置されている事が多いので、Pingやsnmpでは確認できない。
	また、ネットワークの断絶等があった場合、稼働状況を確認できない。
 3.いつ稼働していていつ稼働していなかったのか管理するのが大変
  いつ稼働していていつ稼働していなかったのかがわからないと、データを正確に分析する事が出来ない。

稼働状況の監視については、IoTサービスで行うことがある程度可能だが、サービス自体に手を加える必要があるため、開発のコストが高くなる。

その中で、私は、IoTデバイスの状態監視に着目した。
%IoTデバイスの状態の監視が何故必要なのか
%IoTデバイスの状態監視の難しさ
%既存の解決策と問題点
このような解決策がとられているが、問題が多い

\section{研究の目的}
そこで、IoTサービスとは独立したIoTデバイスの監視サービスを開発することにより、これらの問題を解決できるのではないかと考えた。
本研究では, IoTデバイスの監視サービスを開発することで、IoTデバイスの状態監視を簡単化することを目的とする。

