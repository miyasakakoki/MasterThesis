本章では、研究の背景及び現状の課題について記述し, 本研究の目的について述べる.
\section{研究の背景}
近年, 半導体技術の進歩により, コンピューターの小型化・低価格化が進んでいる. 

%RaspberryPi 123の価格と性能について
%ESP-WROOM-02
%Ardiuno
%beaglebone
%Intel-Edison

さらに, 家庭へのインターネットの普及により, 全ての物がインターネットに接続し相互に情報を交換し合い様々な自動化を実現するIoTが注目されている.\\

%IoTのサービスがいくつか開発されている
%例えば、
%バスの運行情報の掲示
%農業の自動化

このように、IoTサービスの開発が盛んに行われている。
\section{問題}%問題
しかし、IoTサービスの開発・運用において次のような物が問題になっている

%IoTサービスの開発・運用において、このような問題がある。
%セキュリティーの問題
%開発コストの問題
%どれがどれだかわからない、どこに設置したのかわからない問題。
%デバイスの状態の監視
 %状態の監視とは?


その中で、私は、IoTデバイスの状態監視に着目した。
%IoTデバイスの状態の監視が何故必要なのか
%IoTデバイスの状態監視の難しさ
%既存の解決策と問題点
このような解決策がとられているが、問題が多い

\section{研究の目的}
そこで、IoTサービスとは独立したIoTデバイスの監視サービスを開発することにより、これらの問題を解決できるのではないかと考え、
本研究では, IoTサービスの運用・維持におけるデバイスの稼動状態の監視を簡略化することを目的とする.

