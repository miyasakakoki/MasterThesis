\section{研究の背景}
近年,IoTが注目を集めている.
IoTとは,様々なモノがインターネットにつながり,相互に情報をやり取りすることで,利便性の向上を図る概念である.
近年の半導体技術の進歩により,コンピュータが安価・小型になったこと,インターネットへの通信が様々な場所で安価に行えるようになったことにより,注目が集まっている.
IoTの具体的な例としては,建設重機の盗難防止,プリンタのトナー発注自動化,体温や脈拍等の収集・可視化等が挙げられる.
\medskip

IoTサービスとは,IoTによる利便性を顧客に提供するサービスであり,より生活に身近なサービスの登場が期待されている.
上記の例では,プリンタのトナー発注自動化がIoTサービスにあたる.
IoTサービスは,IoT機器とサーバがインターネットを介して通信し合うことで成り立っている.
IoT機器は,モノにコンピュータが取り付けられた物で,周囲の状況を検知,または,周囲へ働きかける機能を持つ.
サーバは,IoT機器からの情報を蓄積・分析し,IoT機器へ指示を送るか,ユーザへ分析結果を表示する機能を持つ.
これらIoT機器とサーバが連携することで,IoTサービスは利便性をユーザへ提供している.
\medskip

このようなIoTサービスを円滑に提供するためには,IoT機器・インターネットへの接続・サーバの状態を監視し,必要に応じて,機器の交換等を行う必要がある.

\section{IoTサービスの維持における課題}
しかし,IoT機器の監視は,技術的課題から既存手法の適用が難しい.
既存手法は,機器の監視のためには,監視サーバと機器の両方に設定が必要となる.
IoT機器が多量に使用されること,IoT機器の追加・撤去・交換は頻繁にあることから,監視サーバへの登録やIoT機器への個別の設定が,大きな負担となっている.
\medskip

IoTサービスの提供者として,太陽光発電発電に係る機器の監視サービスを提供している株式会社ルナネクサスが挙げられる.
株式会社ルナネクサスとは,大阪にある組み込み機器メーカーである.
株式会社ルナネクサスでは,太陽光発電事業を展開している事業主に対し,発電に係る機器の状態や,発電量等を可視化できるサービスを展開している.
しかし,IoT機器の数に増減があることや,故障などによる交換により次のような事が問題となっている.
\begin{itemize}
\item 各IoT機器の状態を監視するために,多数のIoT機器へ個別の設定をしなければならないこと
\item 増減や交換の度に,機器監視システムへの登録をしなければならないこと
\item 交換の為に現地に行くも,類似の機器が多数存在し,外観から交換対象機器がわかりづらいこと
\end{itemize}

\section{研究の目的}
そこで,本研究では,増減や交換の度に監視サービスに登録を行わなければいけない負担を軽減するために,監視サーバにて設定の一元的な管理を提案し,システムを作成する.
監視サービスは、機器の状態や機器名等と紐付いたIDを発行し、ユーザは機器に対しIDを設定する。
機器はIDと紐付いたURLへ、自身の状態を通知する。
管理者は、機器へIDの貼り付けを行う.
\medskip

前項に挙げたIoT機器の監視の為の管理が負担である問題は,機器監視サーバと各IoT機器に設定が分散しているのが原因であると分析し,
機器監視サーバでの設定の一元管理と登録の簡略化,各IoT機器への個別の設定の簡略化により,負担を軽減することを目的とする.

\section{本論文の構成}
第2章では,IoTサービスの維持に関する背景と,IoT機器の監視・管理に関する問題を述べる.
第3章では,既存手法を適用した際の問題点を述べる。
第4章では,第3章で述べた既存手法の問題点から,IoT機器の監視・管理はどうあるべきか提案を述べる.
第5章では,IoT機器監視・管理サービスの実装の詳細について述べる.
第6章では,実験によりIoT機器監視・管理サービスがもたらす効果を検証し,考察を述べる.
第7章では,本研究に関する評価について述べる.
第8章では,本研究を通して得られた知見や今後の課題について述べる.


