作成したシステムを検証するために以下のような実験を行った。
%なんか適当に実験をする
\section{実験概要}
	学内にてWifiプローブパケットを利用した人流観測の実験を再現する
\subsection{実験目的}
	作成したシステムがIoTサービスの開発にどう影響するのか観測する。
\subsection{実験方法}
	第3者にWifiプローブパケットを利用した人流観測のサービスを開発してもらい、その様子を観測する。
	また、観測後にインタビューを行い、システムを利用した場合としない場合の違いについて聞く。
\section{準備}
	RaspberryPi,Wifiドングル,プローブパケット観測ソフトウェア,監視用エージェントを被観測者に提供する。
	被観測者は、それらを用いて、学内の各階に誰が居るのか、また、人の移動の様子を可視化するサービスを構築してもらう。
\section{経過}
\section{考察}


実験により、次のような評価を得ることができた。
%なんか適当に評価を書く

評価から、有効であると分かった。

