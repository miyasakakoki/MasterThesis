本システムの機能検証の為に、ユーザーテストを行った。
本章では、ユーザーテストの結果を報告し、考察する。

\section{シナリオ}
ユーザーテストは、3台のIoT機器で成り立っているIoTサービスを想定し行った。\\
IoT機器には、RaspberryPiとIntelEdisonが使用されている。\\
想定したIoTサービスは、各IoT機器から気温を集め、各地点の気温のグラフを描画するものとした。

本検証では、従来手法による監視の手間と、本サービスによる監視の手間を比較するため、下記項目を重点的に行った。
比較対象として、TelegrafとInfluxdb、Grafanaを用いた機器状態の監視システムを挙げる。

順に結果を報告する。
\section{IoT機器の追加から監視まで}
従来は次の手順を踏むことで、機器の状態を可視化することができた。
\begin{enumerate}
\item とりあえず
\item あとで書く
\end{enumerate}

本システムでは次の手順を踏むことで機器の状態を可視化する事ができる。
\begin{enumerate}
\item ここも
\item あとで書く
\end{enumerate}

比較すると、従来手法ではあった設定ファイルを開き、設定を入力する手順がなくなり、
代わりに、自動起動の為の設定ファイルを開き、設定する手順が増えている。

\section{IoT機器の状態の監視}
従来は次のように機器の状態の監視を行っていた。
\begin{enumerate}
\item とりあえず
\item あとで書く
\end{enumerate}

本システムでは、次の様に機器の状態を監視する。
\begin{enumerate}
\item ここも
\item あとで書く
\end{enumerate}

比較すると、〇〇が楽になっていることが分かる。

\section{過去の機器状態の閲覧}
従来は次のようにこれをするために、次のように行っていた。
\begin{enumerate}
\item とりあえず
\item あとで書く
\end{enumerate}

本システムでは、次のように機器状態の閲覧ができる。
\begin{enumerate}
\item ここも
\item あとで
\end{enumerate}


\section{考察}
本ユーザーテストの結果と、従来の手法を用いた場合の比較を以下に示す。


\section{今後の課題}
ユーザーテストから本システムは、ある程度有効であることが分かったが、以下の点について課題があることが分かった。
\begin{itemize}
\item 機器への設定について\\
	設定ファイルの編集等の手間は削減されたが、エージェントプログラムをIoT機器にインストールするのに手間がかかっていることが分かった。
	従来手法を用いて機器の監視をする際は、自動起動の設定を行う必要はなかったが、本プログラムでは必要としている。
	自動起動の為の設定ファイルを自動生成するか、あるいはエージェントプログラム配布の際、簡単なインストールスクリプト等を一緒に配布することで、大きな手間の削減になると感じでいる。
\item ユーザーインターフェースについて\\
	現在、過去の機器状態の記録については、文字記録として表示しているが、ブラフ表示等の方が見やすいと感じた。
	また、期間を指定して閲覧できる機能も必要であることが分かった。
\end{itemize}


