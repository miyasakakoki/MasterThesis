本システムの機能検証の為に,ユーザーテストを行った.
本章では,ユーザーテストの結果を報告し,考察する.

\section{シナリオ}
ユーザーテストは,3台のIoT機器で成り立っているIoTサービスを想定し行った.\\
IoT機器には,RaspberryPiとIntelEdisonが使用されている.\\
想定したIoTサービスは,各IoT機器から気温を集め,各地点の気温のグラフを描画するものとした.

本検証では,従来手法による監視の手間と,本サービスによる監視の手間を比較するため,下記項目を重点的に行った.
ここでは,従来の手法として,TelegrafとInfluxdb,Grafanaを用いた機器状態の監視システムを比較対象に挙げる.

順に結果を報告する.
\section{機器監視サーバーの構築}
従来の手法では,機器監視をするために,次の様な手間があった.
\begin{enumerate}
\item 機器監視の為にサーバーを用意する
\item 機器監視サーバへInfluxdbをインストールする
\item 機器監視サーバーへGrafanaをインストールする
\item Influxdbに対し,Telegrafとの通信に用いるインターフェースの設定を行う
\item Grafanaに対し,Influxdbと連携するための設定をする.
\end{enumerate}

本サービスを用いる場合,これら準備は不要となる.

\section{各IoT機器の状態を可視化する}
従来の手法ではIoT機器の状態を可視化するまでに以下の手順を踏んでいた.
\begin{enumerate}
\item IoT機器にTelegrafをインストールする
\item IoT機器にインストールされたTelegrafの接続先の設定を行う
\item 各IoT機器固有のIDをTelegrafに設定する.*
\item Grafanaにログインする.
\item Grafanaへ,現在の状態を表示する為のデータベースクエリを入力する.
\item Grafanaへ,過去の状態を表示するためのデータベースクエリを入力する.
\item 各描画パネルに対し,機器名を設定する.
\end{enumerate}

本サービスでは次の手順を踏むことで機器
\begin{enumerate}
\item 本サービスにログインする
\item 本サービスの機器追加ボタンを押し,機器IDをメモ(コピー)する.
\item 本サービスに対し,機器名と機器詳細を入力する.
\item IoT機器にエージェントプログラムをインストールする.
\item IoT機器にエージェントプログラムを自動で起動するよう設定する.
\end{enumerate}

比較すると,Grafanaへデータベースクエリを入力する作業が,なくなっている.
代わりに,自動起動の為の設定ファイルを設定する手順が増えている.

\section{多種多様な機器への移植性が高い}
本サービスで提供しているエージェントプログラムは,その他のパッケージに依存しない単純な構成であるため,移植性が高い.


\section{考察}
ユーザーテストから本システムは,ある程度有効であることが分かったが,以下の点について課題があることが分かった.
\begin{itemize}
\item 機器への設定について\\
	設定ファイルの編集等の手間は削減されたが,エージェントプログラムをIoT機器にインストールするのに手間がかかっていることが分かった.
	従来手法を用いて機器の監視をする際は,自動起動の設定を行う必要はなかったが,本プログラムでは必要としている.
	自動起動の為の設定ファイルを自動生成するか,あるいはエージェントプログラム配布の際,簡単なインストールスクリプト等を一緒に配布することで,大きな手間の削減になると感じでいる.
\item ユーザーインターフェースについて\\
	現在,過去の機器状態の記録については,文字記録として表示しているが,ブラフ表示等の方が見やすいと感じた.
	また,期間を指定して閲覧できる機能も必要であることが分かった.
\end{itemize}


