実験から抽出した要件に基づいて、システムを作成した。
要件は以下のとおりである
\begin{itemize}
	\item 機器の現在の状態を一覧して確認することができる\\
		機器の状態については、機器が稼働していない・機器が稼働している・機器が稼働しているがネットワークに繋がっていないの3つ。
	\item 機器の過去の状態を時刻と共に確認・取得することができる\\
		各機器の状態について、いつからいつまで稼働していたのか、容易に確認することができる必要がある。
	\item 機器の停止、再起動を行うことができる\\
		回収に備えて、機器の停止、再起動を行う事ができると良い。
\end{itemize}

\section{システムの構成}
システムの構成は以下のとおりである。
\subsection{想定するIoT機器}

\subsection{監視エージェントの動き}
監視対象となるIoT機器には、開発した監視エージェントが入っているものとし、起動時に自動でエージェントプログラムを起動するよう設定されているものとする。


ログファイルの中には、過去のシーケンス番号が入っている。

監視エージェントは、起動後、ログファイルがあるか確認し、あった場合、ログファイルから過去のシーケンス番号を読み出す。
無かった場合、過去のシーケンス番号を0にする。

その後、現在のシーケンス番号を0にする。


監視エージェントは、監視サービスに対し過去のシーケンス番号と現在のシーケンス番号、自身が現在正常である旨のメッセージを送信する。
監視サービスから返答があった場合、受理されたとみなし、現在のシーケンス番号を0にする。
その後、過去のシーケンス番号とログファイルを削除する。
また、返答に停止・再起動のコマンドが含まれていた場合、そのコマンドを実行し終了する。
返答が無かった場合、ネットワークに障害があったとみなし、シーケンス番号をインクリメントし、ログファイルに保存する。
この動きを約1分ごとに繰り返す。

\subsection{サービスの動き}
\subsubsection{監視エージェントからメッセージを受け取った時の動き}
監視エージェントからメッセージを受け取った場合、
DBの末尾がコマンドを示すものであった場合、そのコマンドを変数に記録しておく。
過去のシーケンス番号の数だけ、接続されていなかった旨を、最後に通信があった時刻からDBに格納し、
現在のシーケンス番号の数だけ、接続されていなかった旨を、現在の時刻からさかのぼってDBに格納する。
また、現在の時刻にて機器の状態が正常であった旨をDBに格納する。
最後に、コマンドが格納されている変数の内容と、受理した旨のメッセージを監視エージェントに対し送信する。

\subsubsection{ブラウザから入力を受けた場合の動き}
まず、クッキーからセッションを読みだし、ログインしていないユーザーであった場合、ログインページヘと誘導する。
\begin{itemize}
	\item 現在の機器の状態の確認を求められた場合\\
		サービスは、DBから、機器の現在の状態とその他の機器情報を取得し、返却する。
	\item 機器情報の追加・変更・削除を求められた場合\\
		サービスは、DBへ変更を保存する。
	\item ログ一覧を求められた場合\\
		サービスは、DBから機器の状態を取得・整理し、返却する
\end{itemize}

\section{想定するユーザーの定義}
ユーザーは、IoTサービスを構築しようとしているSIさんと定義する。
\section{ユーザーの動き?}
ユーザーの動きは以下のとおりである。
\begin{itemize}
\item デバイスを追加するとき
	\begin{enumerate}
		\item ブラウザからサービスにアクセスし、ログインする。
		\item 画面から追加ボタン(+ボタン)をクリックし、デバイス名、デバイスの説明を入力する。\\
			この際、デバイスIDがすでに決まっているのであれば、それも入力する。
		\item デバイスIDをメモする。(既に決まっているのであれば、特段シなくても良い)
		\item Addボタンを押す
		\item デバイスに対し、エージェントプログラムをインストールし、自動で起動するよう設定する。\\
			その際、エージェントへの引数として、サーバーのIPアドレス、デバイスIDを設定する。
	\end{enumerate}
\item デバイスを削除するとき
	\begin{enumerate}
		\item ブラウザからサービスにアクセスし、ログインする。
		\item 画面から該当のデバイスをクリックし、削除ボタンをクリックする。
		\item 「ほんとに削除するの?」というダイアログが出るので、OKを押し、画面からデバイスが削除されたことを確認する。
	\end{enumerate}
\item 登録されているデバイス情報を変更するとき
	\begin{enumerate}
		\item ブラウザからサービスにアクセスし、ログインする。
		\item 画面から該当のデバイスをクリックし、変更ボタンを押す。
		\item デバイス作成時と同じようなダイアログが表示されるので、デバイス名やデバイス情報を編集する。
		\item OKボタンをクリックし、デバイスの情報が変わったことを確認する。
	\end{enumerate}
\item デバイスの現在の状態を確認するとき
	\begin{enumerate}
		\item ブラウザからサービスにアクセスし、ログインする。
		\item 該当のデバイスを確認する。\\
			緑が稼働している状態、赤が稼働していない若しくは、稼働しているかわからない状態である。
	\end{enumerate}
\item デバイスの過去の状態を確認するとき
	\begin{enumerate}
		\item ブラウザからサービスにアクセスし、ログインする。
		\item Logページボタンを押す。
		\item デバイスのログ一覧が出るので、該当デバイスを探しだし、確認する。
	\end{enumerate}
\end{itemize}









