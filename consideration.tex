検証では,機器IDとURLの組をサーバ側で管理することで,機器からの通知に基づいて監視可能であることを確認した.
検証では,サーバへの機器の登録,機器へのIDの付与は手動で行ったが,機器からの通知による監視,サーバ上でURLとIDの組み合わせの管理について,確認を行った.
これにより,機器の追加・移動の際に,機器への設定の手間や,機器が接続されるネットワークの調査等の手間を省くことができた.
また,機器へのID付与や機器の登録の作業が負担となったので,実装を進め,省力化を図り検証を行う事を考えている.

\begin{itemize}
\item サーバから機器IDを付与することで,機器への設定を簡略化し,負担を軽減する事
\item IoT機器の変更や追加の際に,サーバに対し登録を行わなければならない負担が軽減された事
\end{itemize}

今後の課題として,次があげられる.
\begin{itemize}
\item IoT機器のセキュリティ\\
	現状,IoT機器から監視サーバまでの間は,SSLを用いることで暗号化している.
	しかし,IoT機器自体を分解することで,機器IDを取得し,機器のなりすましが行われる危険がある.
	そのため,なんらかの方法で機器へのなりすましを検知し通知することや,機器に耐タンパ性を持たせる必要があるように考える.
\item 物理との紐付けの改良\\
	現状では,機器に機器IDを貼り付ける事で解決しているが,現地に行った際にどの機器IDが交換の必要があるのか,見比べなくてはならない.
	数十桁の機器IDを一瞬で判別することは困難であり,改良の余地があるように思う.
	具体的には,QRコードを用いて,スマートフォン等から監視サービスの特定の機器の状態閲覧画面にアクセスすることができるようにする等が必要である.
	その際には,セキュリティの観点から,一般の利用者へ対する説明等の表示と,機器の管理者に対する表示を分ける必要が出てくると考えている.

	また,現状では,一旦サービスに登録されたことを確認してからラベリングとしているが,QRコードを使用した場合,機器に対して予めラベリングを施しておくことでより設定の手間が省けるように感じている.
	具体的な実装としては,監視サービス側にてQRコードをまとめて発行し,ユーザは,QRコードを電源を入れる順番に貼り付けていく.
	機器側は,アクセスのあった順に発行したQRコードに埋め込まれた機器IDを割り振っていく.
	これにより,物理的なヒモ付の手間や,交換の際の手間等がより省けると感じている.
\item  通知間隔の問題\\
	機器から監視サーバへの通知は,現状では1分おきに行っているが,各種IoT機器にてこの1分が妥当なものなのか検証する必要があるように感じている.
	何故ならば,省電力を目的として,機器自体のスリープや,ネットワークからの離脱等を行う場合があるためである.
	このような場合,監視サーバにて通知間隔の設定を行うことで,誤認等を防ぐ必要がある.
	また,該当の機器の状態が不安定である場合など,意図的に監視間隔を短く設定することもできるよう考慮する必要があると考える.
\item 機器の操作\\
	RaspberryPi等,IoT機器によっては,電源が急に抜けた場合,記録していたデータが失われてしまう物がある.
	そのため,撤去の際に電源を切る操作を行ってから,電源を切る必要がある.
	しかし,多くのIoT機器は,画面やキーボード等はついておらず,現地に画面やキーボードを運び接続するか,遠隔から操作する他無い.
	また,現地にエンジニアを派遣する必要もある.
	この手間を解決するために,機器の監視サービスの機能として,電源のOFF・再起動ボタンや,遠隔からのログイン等を行えるようにする必要も感じる.
\item IoTサービスとの連携\\
	IoTサービスは,収集したデータの分析や可視化を行う.
	そのため,IoT機器が現在稼働しているのか,過去に置いて稼働していたのか,等を知る必要がある.
	現状では,分析などの際に手動で入力することや,ある程度の誤差を容認することで解決しているが,
	機器監視サービスが,監視情報をIoTサービスに対して提供することでも解決できるのではないかと考えている.
	そのために,監視サービスがIoTサービスとのインターフェースを持つ必要があると考えている.
\end{itemize}

\begin{comment}
 ・IoT機器自体のセキュリティの問題(機器が盗まれた場合や,機器から機器IDを取られた場合について)
 ・物理との紐付けの改良(QRコード等を使用し,監視サービスと連携すること)
 ・監視間隔の問題(妥当な間隔なのかどうか)
 ・機器の操作(機器へログインする,電源のON・OFFを行うこと)
 ・IoTサービスとの連携(IoTサービスに対してAPIを提供する等)
 ・機器IDの生成(とりあえずハッシュとして設計したが,どの程度の衝突耐性が必要となるのか調べる必要がある,類推されないIDが望まれる)



まだかけていない.

書くことは,問題として認識した3つの事を主軸にこの手法が有効である事を言う.
また,今後どのような拡張等が見込まれるのかを書く.
\end{comment}
