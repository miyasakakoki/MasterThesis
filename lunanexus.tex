
IoT機器の監視が重要であること、監視にて困っている事、また監視システムには何が必要なのかを探るため、太陽光発電所へIoT機器を販売している株式会社ルナネクサス様を訪問した。

株式会社ルナネクサス様では、太陽光発電事業を展開している事業主に対し、発電量や発電に係る機器の状態を遠隔から監視できるIoTサービスを展開している。
独自に開発したIoT機器を発電に係る機器に取り付け、SORACOM Airというインターネット接続サービスを使用して、機器からの情報をサーバーへ蓄積し、ユーザーへ提供している。
SORACOM Airとは、後術するIoT機器向けのインターネット接続サービスの事である。

太陽光発電に係る機器は、通常発電所のわきに小さなプレハブを建て、その中に設置する。
太陽光発電所は日当たりが良いため、夏場、プレハブの中は太陽の熱と機器が発する熱で60度を超える。
そのため、機器が正常に動作しているかどうかを監視する必要がある。

このように、株式会社ルナネクサス様でも、監視について問題を持っていることが分かった。
