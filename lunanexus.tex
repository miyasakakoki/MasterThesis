
IoT機器の監視が重要であること、監視にて困っている事、また監視システムには何が必要なのかを探るため、太陽光発電所へIoT機器を販売している株式会社ルナネクサスを訪問した。
株式会社ルナネクサスは、大阪にある組み込み機器メーカーで、
太陽光発電所事業を行っている事業主に対し、IoT機器を開発・提供し、発電量や発電に係る機器の異常を検知し通知するサービスを行っている。

IoT機器は、SORACOM Airの回線を通してインターネットにつながっている。
IoT機器は、発電機器の異常や発電量をインターネット上のサーバーに送信し、サーバはユーザーに対し通知を行う。

IoT機器は、発電機器に接続されるため、同じ場所に設置される場合が多い。
しかし、発電機器は、発電所の屋外のプレハブの中に設置されることが多く、夏場は太陽の熱と機器が発する熱で高温になる。
そのため、IoT機器自身の不調も多い。


VPNの話を書きたい!








株式会社ルナネクサス様では、太陽光発電事業を展開している事業主に対し、発電量や発電に係る機器の状態を遠隔から監視できるIoTサービスを展開している。
独自に開発したIoT機器を発電に係る機器に取り付け、SORACOM Airというインターネット接続サービスを使用して、機器からの情報をサーバーへ蓄積し、ユーザーへ提供している。
SORACOM Airとは、後術するIoT機器向けのインターネット接続サービスの事である。

太陽光発電に係る機器は、通常発電所のわきに小さなプレハブを建て、その中に設置する。
太陽光発電所は日当たりが良いため、夏場、プレハブの中は太陽の熱と機器が発する熱で60度を超える。
そのため、機器が正常に動作しているかどうかを監視する必要がある。

このように、株式会社ルナネクサス様でも、監視について問題を持っていることが分かった。
