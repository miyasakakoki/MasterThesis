\documentclass[a4paper]{jreport}
\usepackage[top=30truemm,bottom=30truemm,left=30truemm,right=30truemm]{geometry}


\pagestyle{plain}
\begin{document}
\makeatletter

\title{IoTサービスにおけるIoTデバイス監視簡単化サービスの提案}
\author{15006・宮坂 虹槻}
\date{2017年 1月 28日}
\def\@teacher{横山 輝明}


\newgeometry{left=2cm,bottom=2cm,top=4cm,right=2cm}
\begin{titlepage}\begin{center}
\thispagestyle{plain}
{\Huge \textbf{修士論文} \par}
\vspace{1.5cm}
{\LARGE\gt 題目 \par}
{\LARGE\gt \underline{\@title} \par}
\vspace{2.5cm}
{\LARGE\gt 学籍番号・氏名 \par}
\vspace{1.5cm}
{\LARGE \underline{\@author} \par}
\vspace{1.5cm}
{\LARGE\gt 指導教員 \par}
\vspace{1.5cm}
{\LARGE\gt \underline{\@teacher} \par}
\vspace{1.5cm}
{\LARGE\gt 提出日 \par}
\vspace{1.5cm}
{\LARGE\gt \underline{\@date} \par}
\vspace{1.5cm}
{\Large\gt
神 戸 情 報 大 学 院 大 学\\
情報技術研究科  情報システム専攻\\
\par}
\end{center}\end{titlepage}
\restoregeometry
\makeatother

\tableofcontents

\begin{abstract}
%	背景
 半導体技術の進歩 -> コンピューターの小型化・低価格化 1
 インターネット回線網の普及 2
 1&2 -> IoTが注目されている

 IoTとは
 Internet of Things
 様々な物がインターネットにつながり、相互に情報を交換し合うことで、様々な自動化を実現しよう という考え方。
 IoTの例:
1.
2.
3.

 また、IoTが注目される事によって、その自動化で収益を得ようとするサービスが登場し始めた。
 IoTサービスの例:
 1.餌やりの自動化
 2.
 3.

%問題
 しかし、IoTサービスを開発・運用するには、様々な問題がある。
 1.開発が大変とか?
 2.故障検知の問題とか?(死活監視)
	死活監視が何故必要か->サービスを止めないために必要
	死活監視が
 3.設置場所の問題とか?

その中で、私は、デバイスの故障検知?の問題に目をつけた。
何故そこに目をつけたのか
問題の詳細

このように、デバイスの故障検知?には、上記のような問題が複雑に絡み合っており、単純には解決できない。

% 目的
そこで、IoTサービスとは独立したIoTデバイスの監視サービスを開発することにより、デバイスの故障検知に係る問題の解決を図ることにした。
システムの構築に先立って、どのような機能が必要となるのか実験し、次のような機能が必要になることが分かった。
1.
2.
3.
また、必須ではないものの、次のような機能があると、嬉しい事が分かった。
1.
2.
3.
%	構成
これら必要な機能を踏まえ、次のようなシステムを構築
%	検証と結果
システムを構築し協力を得て、検証を行った所、次のような結果が得られた。

これら結果により、システムの有効性が立証できた。
%	今後の課題
今後の課題としては、
1.
2.
3.
\end{abstract}

\chapter{序論}
本章では、研究の背景及び現状の課題について記述し, 本研究の目的について述べる.
\section{研究の背景}
近年, 半導体技術の進歩により, コンピューターの小型化・低価格化が進んでいる. 

%RaspberryPi 123の価格と性能について
%ESP-WROOM-02
%Ardiuno
%beaglebone
%Intel-Edison

さらに, 家庭へのインターネットの普及により, 全ての物がインターネットに接続し相互に情報を交換し合い様々な自動化を実現するIoTが注目されている.\\

%IoTのサービスがいくつか開発されている
%例えば、
%バスの運行情報の掲示
%農業の自動化

このように、IoTサービスの開発が盛んに行われている。
\section{問題}%問題
しかし、IoTサービスの開発・運用において次のような物が問題になっている

%IoTサービスの開発・運用において、このような問題がある。
%セキュリティーの問題
%開発コストの問題
%どれがどれだかわからない、どこに設置したのかわからない問題。
%デバイスの状態の監視
 %状態の監視とは?


その中で、私は、IoTデバイスの状態監視に着目した。
%IoTデバイスの状態の監視が何故必要なのか
%IoTデバイスの状態監視の難しさ
%既存の解決策と問題点
このような解決策がとられているが、問題が多い

\section{研究の目的}
そこで、IoTサービスとは独立したIoTデバイスの監視サービスを開発することにより、これらの問題を解決できるのではないかと考え、
本研究では, IoTサービスの運用・維持におけるデバイスの稼動状態の監視を簡略化することを目的とする.

\chapter{既存の解決策とその課題}

\chapter{提案する解決策}
既存の解決策では、こんなものが問題となっていた。
また、次のような希望があることも分かった。

以上までで述べた問題点をまとめると、下記のようになる。

設計
1.xxxxx
2.xxxxx

\chapter{提案システムの仕様と構成}

\chapter{提案システムの実装}

\chapter{検証と評価}

\chapter{考察}
\chapter{結論}
\chapter{謝辞}
\chapter{参考文献}

\end{document}
