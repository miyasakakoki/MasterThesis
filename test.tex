\documentclass[a4paper]{jreport}
\usepackage[top=30truemm,bottom=30truemm,left=30truemm,right=30truemm]{geometry}


\pagestyle{plain}
\begin{document}
\makeatletter

\title{IoTサービスにおけるIoTデバイス監視簡単化サービスの提案}
\author{15006・宮坂 虹槻}
\date{2017年 1月 28日}
\def\@teacher{横山 輝明}


\newgeometry{left=2cm,bottom=2cm,top=4cm,right=2cm}
\begin{titlepage}\begin{center}
\thispagestyle{plain}
{\Huge \textbf{修士論文} \par}
\vspace{1.5cm}
{\LARGE\gt 題目 \par}
{\LARGE\gt \underline{\@title} \par}
\vspace{2.5cm}
{\LARGE\gt 学籍番号・氏名 \par}
\vspace{1.5cm}
{\LARGE \underline{\@author} \par}
\vspace{1.5cm}
{\LARGE\gt 指導教員 \par}
\vspace{1.5cm}
{\LARGE\gt \underline{\@teacher} \par}
\vspace{1.5cm}
{\LARGE\gt 提出日 \par}
\vspace{1.5cm}
{\LARGE\gt \underline{\@date} \par}
\vspace{1.5cm}
{\Large\gt
神 戸 情 報 大 学 院 大 学\\
情報技術研究科  情報システム専攻\\
\par}
\end{center}\end{titlepage}
\restoregeometry
\makeatother

\tableofcontents

\begin{abstract}
\input abstract.tex
\end{abstract}

\chapter{序論}
\input introduction.tex

\chapter{既存の解決策とその課題}
\input survey.tex

\chapter{提案する解決策}
\input proposal.tex

\chapter{提案システムの仕様と構成}

\chapter{提案システムの実装}

\chapter{検証と評価}

\chapter{考察}
\chapter{結論}
\chapter{謝辞}
\chapter{参考文献}

\end{document}
