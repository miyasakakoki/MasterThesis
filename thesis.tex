\documentclass[a4paper]{jreport}
\usepackage[top=25truemm,bottom=20truemm,left=25truemm,right=20truemm]{geometry}
\usepackage[dvipdfmx]{graphicx}
\usepackage{url}
\usepackage{comment}


\pagestyle{plain}
\renewcommand{\abstractname}{内容梗概}
\begin{document}
\makeatletter
\renewcommand{\bibname}{参考文献}

\title{IoT機器からの通知に基づいた機器監視サービスの開発}
\author{15006・宮坂 虹槻}
\date{2017年 1月 28日}
\def\@teacher{横山 輝明}

\newgeometry{left=2cm,bottom=2cm,top=4cm,right=2cm}
\begin{titlepage}\begin{center}
{\Huge \textbf{修士論文} \par}
\vspace{1.5cm}
{\LARGE\gt 題目 \par}
{\LARGE\gt \underline{\@title} \par}
\vspace{2.5cm}
{\LARGE\gt 学籍番号・氏名 \par}
\vspace{1.5cm}
{\LARGE \underline{\@author} \par}
\vspace{1.5cm}
{\LARGE\gt 指導教員 \par}
\vspace{1.5cm}
{\LARGE\gt \underline{\@teacher} \par}
\vspace{1.5cm}
{\LARGE\gt 提出日 \par}
\vspace{1.5cm}
{\LARGE\gt \underline{\@date} \par}
\vspace{1.5cm}
{\Large\gt
神 戸 情 報 大 学 院 大 学\\
情報技術研究科  情報システム専攻\\
\par}
\end{center}\end{titlepage}
\restoregeometry
\makeatother

\pagenumbering{roman}
\tableofcontents
\listoffigures

\begin{abstract}
\input abstract.tex
\end{abstract}

\pagenumbering{arabic}
\chapter{はじめに}
\input introduction.tex

\chapter{IoTサービスの維持における問題}
\input issue.tex

\chapter{IoT機器からの通知に基づく機器監視サービスの提案}
\input proposal.tex

\chapter{機器監視サービスの実装}
\input implementation.tex

\chapter{機器監視サービスのユーザーテストと考察}
\input evaluation.tex

\chapter{おわりに}
\input conclution.tex

\chapter{謝辞}

%\chapter{参考文献}
\input reference.tex

\begin{comment}
流れ
序論
・IoTが流行している。
  ->何故?
	・コンピューターの高性能化と価格の低下 -> ほんとに?
	・家庭へのインターネットの普及 -> ほんとに?
  -> ほんとに?

・IoTのデバイス監視で困っている。
  -> デバイスの監視が本当に必要なのか?
	・サービスを止めないために必要 -> ほんとに?
  -> 何故デバイスの監視で困っているのか?
	・見に行くわけに行かない。
	  ->何故?
		・設置場所が離れていることがある。-> ほんとに?
		・数が多くて見きれない。 -> ほんとに?
	・サービスに組み込むのにも手間がかかる -> ほんとに?
  -> ほんとに?

提案
・IoT機器向けの機器監視サービスを作ったら、楽になるのでは?
  -> 他の解決策はどうなの?何がダメなの?
	・Ping
	・SNMP
	・Influxdb...
	・Elasticksearch...
  -> 他のがダメなのに、コレが何故OKなの?
	・IoTサービスの開発に特化しているため
	  ->どう特化してるの?
		・グラフ作成等の手間が不要
		・一覧して見ることができる
		・機器から通知が送られてくるので、NAPTで突っかからない。
  -> ほんとに?=>検証へ

検証
・田頭さんの研究で使ってもらい評価を得る。
 1. 使い方について簡単に説明
 2. 使っている様子を観測
  3. 最後に聞き取りを実施
-> なんだって?
 ・こんな意見が聞けた。
 ・サービスについては、〇〇だと言われた。
 ・観察している中で、こんな所が気になった。

結論
・確かに楽になった or 楽にならなかった。
 -> 何故?

今後の課題としてこんなことがあった。
 ・
\end{comment}

\end{document}
