\documentclass[a4paper]{jreport}
\usepackage[top=25truemm,bottom=20truemm,left=25truemm,right=20truemm]{geometry}
\usepackage[dvipdfmx]{graphicx}
\usepackage{listings,jlisting}
\usepackage{setspace}
\usepackage{url}
\usepackage{comment}


\lstset{
language={Python},
basicstyle={\small},
identifierstyle={\small},
commentstyle={\small\itshape},
keywordstyle={\small\bfseries},
ndkeywordstyle={\small},
stringstyle={\small\ttfamily},
frame={tb},
breaklines=true,
columns=[l]{fullflexible},
numbers=left,
xrightmargin=0zw,
xleftmargin=3zw,
numberstyle={\scriptsize},
stepnumber=1,
numbersep=1zw,
lineskip=-0.5ex
}



\pagestyle{plain}
\renewcommand{\abstractname}{内容梗概}
\begin{document}
\makeatletter
\renewcommand{\bibname}{参考文献}

\title{IoT機器からの通知に基づいた機器監視サービスの開発}
\author{15006・宮坂 虹槻}
\date{2017年 2月 13日}
\def\@teacher{横山 輝明}

\newgeometry{left=2cm,bottom=2cm,top=4cm,right=2cm}
\begin{titlepage}\begin{center}
{\Huge \textbf{修士論文} \par}
\vspace{1.5cm}
{\LARGE\gt 題目 \par}
{\LARGE\gt \underline{\@title} \par}
\vspace{2.5cm}
{\LARGE\gt 学籍番号・氏名 \par}
\vspace{1.5cm}
{\LARGE \underline{\@author} \par}
\vspace{1.5cm}
{\LARGE\gt 指導教員 \par}
\vspace{1.5cm}
{\LARGE\gt \underline{\@teacher} \par}
\vspace{1.5cm}
{\LARGE\gt 提出日 \par}
\vspace{1.5cm}
{\LARGE\gt \underline{\@date} \par}
\vspace{1.5cm}
{\Large\gt
神 戸 情 報 大 学 院 大 学\\
情報技術研究科  情報システム専攻\\
\par}
\end{center}\end{titlepage}
\restoregeometry
\makeatother

\setstretch{1.5}
\pagenumbering{roman}
\tableofcontents
\listoffigures
\listoftables

\begin{abstract}
\input abstract.tex
\end{abstract}

\pagenumbering{arabic}
%第1章
\chapter{はじめに}
\input introduction.tex
%第2章
\chapter{IoTサービスの監視における問題}
\input issue.tex
%第3章
\chapter{既存の監視手法}
\input survay.tex
%第4章
\chapter{IoT機器からの通知に基づく機器監視サービスの提案}
\input proposal.tex
%第5章
%\chapter{機器監視サービスの実装}
\input implementation.tex
%第6章
\chapter{機器監視サービスの動作テスト}
\input evaluation.tex

\chapter{おわりに}
\input conclution.tex

\chapter{謝辞}
\input acknowledgment.tex

%\chapter{参考文献}
\input reference.tex


\appendix
\section{岡本商店街での事例}
\input okamoto.tex
%\section{ソースコード}
%\input source.tex

\begin{comment}
流れ
背景
・IoTが流行している.
  ->何故?
	・コンピューターの高性能化と価格の低下
	・家庭へのインターネットの普及

問題
・IoTの開発運用において、サービスを止めないためにIoT機器の監視が必要となっている。
 ・岡本商店街では、
  ・設置ミス、設定ミスや電源トラブル、ネットワークからの断絶などにより、記録がとぎれとぎれに ー> 分析の際に困った。 ー>監視が必要
   ー>何故、設置ミスや設定ミスがおきてしまったのか? ー> 外観が似ていたため、別の機器に設定していることに気が付かなかった。
 ・株式会社ルナネクサス訪問
  ・株式会社ルナネクサスとは、大阪の組み込みの会社。太陽光発電のIoTサービスを提供している。
  ・聞き取りをした所、次のような結果が得られた
   ー>IoT機器の監視を必要としている。
   ー>IoT機器に対して設定にて困っている。
 ・監視が必要とされているが、IoT機器の設定や登録で困っている。
  ・IoT機器の数が多い・追加・交換・撤去などが運用時に行われる。しかし、機器の設定を変えるにはエンジニアが必要。
   ー>そのため、エンジニアを頻繁に現地に送り込まなくてはならない
   ー>大量の機器に個別の設定を行うことが大きな負担となっている
  ・物理との紐付けが大変
   -> 似た外観の機器が大量に使用される
  ・既存の監視手法では、解決しづらい
   -> NAPT問題
   -> 追加・交換・移動の度に、サーバーと機器の設定を更新シなければならない。
   -> 頻繁にある交換や追加の度に大量の機器を監視するのには不向き
 なぜ、監視が難しいのか -> サーバーとIoT機器の両方に設定が必要だからではないか?

提案
・IoT機器向けの機器監視サービスを開発し、サーバーにて機器の管理を一元的に管理する
 IoT機器の設定を可能な限り簡略化 -> IoT機器の数に対応
 サーバーにてQRコードを発行 -> 紐付けが難しい問題を解決
 IoT機器から通信をする -> NAPT問題を解決
 サーバーへの登録を自動化 -> 追加や撤去の際の労力の削減

実装


結論
・確かに楽になった or 楽にならなかった.
 -> 何故?

今後の課題としてこんなことがあった.
 ・
\end{comment}

\end{document}
